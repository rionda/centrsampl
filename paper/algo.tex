\section{Algorithm}\label{sec:algo}
In this section we present our algorithm to compute a set of approximations for the
betweenness centrality of all vertices in a graph through sampling, with
probabilistic guarantees on the quality of the approximations.

The intuition behind the algorithm is the following. Given a graph $G=(V,E)$
with vertex-diameter $\Delta_G$ and two parameters $\varepsilon,\delta\in(0,1)$
we first compute a sample size $r$ using~\eqref{eq:vceapprox} with
$d=\lfloor\log_2(\Delta_G-2)\rfloor+1$:
\begin{equation}\label{eq:samplesize}
r=\frac{c}{\varepsilon^2}\left(\lfloor\log_2(\Delta_G-2)\rfloor+1+\ln\frac{1}{\delta}\right)\enspace.
\end{equation}
The sample size is sufficient to ensure to achieve the desired accuracy
(expressed through $\varepsilon$) with the desired confidence (expressed through
$1-\delta$). Then the algorithm repeats, for $r$ times, the following steps:
1.~it samples a pair $u,v$ of distinct vertices uniformly at random, 2.~it
computes the set $\mathcal{S}_{uv}$ of all shortest paths between $u$ and $v$,
3.~it selects a path $p$ from $\mathcal{S}_{uv}$ uniformly at random and
increase by $\binom{n}{2}$ the betweenness counter of each vertex in
$\mathsf{Int}(p)$. Note that if the sampled vertices $u$ and $v$ are not
connected, we defined $\mathcal{S}_{uv}=\{p_\emptyset\}$ so nothing is done in
step 3. In the end, the estimator for the betweenness $\betw(w)$ of a vertex $w$
is 
\[
\tilde\betw(w) = \binom{n}{2}\frac{1}{k}\sum_{p\in S}
\mathds{1}_{\mathsf{Int}(p)}(w) = \binom{n}{2}\frac{1}{k}\sum_{p\in S}
\mathds{1}_{\mathcal{T}_w}(p)\enspace.
\]

There are two crucial steps in this algorithm: the computation of $\Delta_G$ and
the sampling of a path uniformly at random from $\mathcal{S}_{uv}$. We first
deal with the latter, and present an approximation algorithm for $\Delta_G$ in
Sect.~\ref{sec:diam}. Algorithm~\ref{alg:algorithm} presents the
pseudocode of the algorithm, including the steps to select a random path.
The \texttt{getDiameterApprox()} procedure is our approximation algorithm for
computing an approximation for $\Delta_G$ and its pseudocode is presented in
Alg.~\ref{alg:diam}.

\paragraph{Sampling a shortest path}
Our procedure to select a random shortest path from $\mathcal{S}_{uv}$ is
inspired by the dependencies accumulation procedure used in Brandes' exact
algorithm~\citep{Brandes01}. Given a vertex $w$, ~\citet{Brandes01} showed how
to compute, for every vertex $z$, the number $\sigma_{wz}$ of shortest paths
from $w$ to $z$. We can compute this number for every vertex belonging to at
least one path from the sampled vertex $u$ to the sampled vertex $v$ while we
compute the set $\mathcal{S}_{uv}$ of all the shortest paths from $u$ to $v$. We
can use this information to select a shortest path $p$ uniformly at random from
$\mathcal{S}_{uv}$. We assume that $u$ and $v$ are connected, otherwise, the
only possibility is to select the empty path $p_\emptyset$. For each vertex $w$
let $P_u(w)$ be the subset of neighbors of $w$ that are \emph{predecessors} of
$w$ along the shortest paths from $u$ to $w$. Let $p^*=\{v\}$. Starting from $v$,
we select one of its predecessors $z\in P_u(v)$ using a weighted random sampling
where each $z\in P_u(v)$ has probability $\sigma_{uz}/\sum_{w\in
P_u(v)}\sigma_{uz}$ of being sampled. We add $z$ to $p^*$ and  then repeat the
procedure for $z$, selecting one of its predecessors from $P_u(z)$ and adding it
to $p^*$, and so on until we reached $u$. Note that we can update the estimation
of the betweenness of the internal vertices that are added to $p^*$ (the only
ones for which the estimation is updated) as we compute $p^*$.

\begin{lemma}
  The path $p*$ built according to the above procedure is selected uniformly at
  random among the paths in $\mathcal{S}_{uv}$.
\end{lemma}
\begin{proof}
  The probability of sampling $p^*=(u,z_1,\dotsc,z_{|p^*|-2},v)$ equals to the
  product of the probabilities of sampling the vertices internal to $p^*$, hence
  \[
  \Pr(p^*)=\frac{\sigma_{uz_{|p^*|-2}}}{\sigma_{uv}}\frac{\sigma_{uz_{|p^*|-3}}}{\sigma_{uz_{|p^*|-2}}}\dotsb
  \frac{1}{\sigma_{uz_2}}=\frac{1}{\sigma_{uv}}
  \]
  where we used \citep[Lemma3]{Brandes01} which tells us that for $w\neq u$,
  \[
  \sigma_{uw}=\sum_{j\in P_u(w)}\sigma_{uj}
  \]
  and the fact that for $z_1$, which is a neighbor of $u$, $\sigma_{uz_1}=1$.
\end{proof}

\begin{algorithm}[h]
  \SetKwInOut{Input}{Input}
  \SetKwInOut{Output}{Output}
  \SetKwComment{Comment}{//}{}
  \SetKwFunction{VertexDiameter}{getVertexDiameter}
  \SetKwFunction{SamplePair}{sampleUniformVertexPair}
  \SetKwFunction{SampleUniform}{sampleUniformPath}
  \SetKwFunction{PairShortestPaths}{computeAllShortestPaths}
   \DontPrintSemicolon
  %\dontprintsemicolon
  \Input{a graph $G=(V,E)$ with $|V|=n$, real values $\varepsilon,\delta\in(0,1)$}
  \Output{A set of approximations of the betweenness centrality of the vertices
  in $V$}
  \ForEach{$w\in V$}
  {
  $\tilde\betw(v)\leftarrow 0$
  }
  $\Delta_G\leftarrow$\VertexDiameter{G}\label{alg:diamcomp}\; 
  $r\leftarrow (c/\varepsilon^2)(\lfloor\log_2(\Delta_G-2)\rfloor+\ln(1/\delta))$\;
  \For{$i\leftarrow 1$ to $r$}
  {\label{algline:forloop}
  $(u,v)\leftarrow$\SamplePair{$V$}\;
  $\mathcal{S}_{uv}\leftarrow$\PairShortestPaths{$u,v$}\;
  \Comment{Random path sampling and estimation update}
  \If{$\mathcal{S}_{uv}\neq\{p_\emptyset\}$}
  {
  $j\leftarrow v$\;
  $s\leftarrow v$\;
  $t\leftarrow v$\;
  \While{$t \neq u$} {
  sample $z\in P_u(t)$ with probability $\sigma_{uz}/\sigma_{us}$\;
  \If{$z\neq u$} {
  $\tilde\betw(z) \leftarrow \tilde\betw(z)+\binom{|V|}{2}$\;
  $s\leftarrow t$\;
  $t\leftarrow z$\;
  }
  }
  }
  } % end For
  \Return{$\{\tilde\betw(v), v\in V\}$}
  \caption{Computes approximations $\tilde\betw(v)$ of the betweenness
  centrality $\betw(v)$ for all vertices $v\in V$.}
  \label{alg:algorithm}
\end{algorithm}

The algorithm we described offers probabilistic guarantees on the quality of all
approximations of the betweenness centrality.
\begin{lemma}\label{lem:correctness}
  With probability at least $1-\delta$, all the approximations computed by the
  algorithm are within $\varepsilon\binom{n}{2}$ from their real value:
  \[
  \Pr\left(\exists v\in V \mbox{ s.t. }
  |\betw(v)-\tilde\betw(v)|>\varepsilon\binom{n}{2}\right)<\delta\enspace .
  \]
\end{lemma}

\begin{proof}
  For each $p_{uv}\in\mathbb{S}_G$ let
  \[
  \pi_G(p_{uv})=\frac{2}{\binom{n}{2}|\mathcal{S}_{uv}|}\enspace.
  \]
  It is easy to see that $\pi_G$ is a probability distribution and
  $\pi_G(p_{uv})$ is the probability of sampling the path $p_{uv}$ during an
  execution of the loop on line~\ref{algline:forloop} in
  Alg.~\ref{alg:algorithm}.
  
  Consider the range set $\range_G$ and the probability distribution $\prob_G$.
  Let $S$ be the set of paths sampled during the execution of the algorithm.
  For $r$ as in~\eqref{eq:samplesize}, the sample $S$ is a
  $\varepsilon$-approximation to $(\range_G,\prob_G)$ with probability at least
  $1-\delta$. Suppose that this is indeed the case, then from
  Def.~\ref{def:eapprox} and the definition of $\range_G$ we have that
  \[
  \left|\prob_G(\mathcal{T}_v) - \frac{1}{k}\sum_{p\in
  S}\mathds{1}_{\mathcal{T}_v}(p)\right|=\left|\prob_G(\mathcal{T}_v) -
  \frac{1}{\binom{n}{2}}\tilde\betw(v)\right|\le\varepsilon, \forall v\in
  V\enspace.
  \]
  From the definition of $\prob_G$ we have
  \[
  \prob_G(\mathcal{T}_v)=\frac{1}{\binom{n}{2}}\sum_{p_{uw}\in\mathcal{T}_v}\frac{1}{|\mathcal{S}_{uw}|}=\frac{1}{\binom{n}{2}}\betw(v),
  \]
  which concludes the proof.
\end{proof}

\paragraph{Unique shortest paths case} When the graph $G$ is undirected and
there is an unique shortest path between each pair of node, then one can apply
Lemma~\ref{lem:vcdimuppboundunique} and obtain a smaller sample size
\[ k= \frac{c}{\varepsilon^2}\left(3+\ln\frac{1}{\delta}\right)
\]
to approximate the betweenness centralities of all the vertices. Unique shortest
paths are common in or even enforced in road networks~\citep{GeisbergerSS08} by
slightly perturbing the edge weights or having a deterministic tie breaking
policy.

\subsection{Approximating the diameter}\label{sec:diam}
The algorithm presented in the previous section requires the knowledge (or the
computation) of the vertex-diameter $\Delta_G$) of the graph $G$ (line
\ref{alg:diamcomp} of Alg.~\ref{alg:algorithm}). Computing $\Delta_G$ exactly
can be done by finding the shortest paths between all pair of vertices in $V$
and computing the maximum size. This can be done in time \XXX. Such an exact
computation would defeat our purposes, because once we have all the shortest
paths, we can compute the betweenness of all the nodes exactly. Given that
Thm.~\ref{thm:eapprox} only requires an upper bound to the VC-dimension of the
range set, an upper bound to the vertex-diameter would be sufficient for our
purposes, provided its computation is fast and it does not result in a much
larger sample size than if we used the exact value for $\Delta_G$. There are
known algorithms to approximate the diameter of a
graph~\citep{AingwordCIM99,BoitmanisFL06,RodittyW12}, with various running times
and quality of approximations.

\XXX TBD: This can actually be extended to directed case as well. We just need
to also compute the ``backwards'' shortest paths \emph{to} the sampled nodes, and
choose the maximum of these ``incoming'' paths and the maximum of the
``outgoing'' paths. \MR

\XXX TBD: take care of the weighted case. Say that we can just use $n$ as
approximation and be happy with it. Experiments should show that we are still
faster than Brandes and Pich.

\XXX TBD: mention that we need to compute the connected components.

For our case, if $G$ is \emph{undirected} and \emph{all the weights are
unitary}, a $2$-approximation to $\Delta_G$ is sufficient and can be computed in
time \XXX by selecting a vertex $v\in V$ uniformly at random, computing the
shortest paths from $v$ to all other vertices in $V$, and taking
$\tilde\Delta_G$ to be the sum of the sizes of the two longest shortest paths
from $v$ to two distinct other nodes $u$ and $w$. 

\begin{lemma}\label{lem:diam}
  $\Delta_G\le\tilde\Delta_G\le 2\Delta_G$.
\end{lemma}
\begin{proof}
  Let $p_{vu}$ be the shortest path between $v$ and $u$ and analogously
  $p_{vw}$. We have $\tilde\Delta_G\le 2\Delta_G$ because
  $|p_{vu}|,|p_{vw}|\le\Delta_G$, so $|p_{vu}|+|p_{vw}|\le 2\Delta_g$. To see
  that $\tilde\Delta_G\ge\Delta_G$, consider a pair of nodes $x$ and $z$ such
  that the size of a shortest path between $x$ and $z$ is equal to $\Delta_G$.
  Let $p_{xv}$ be a shortest path between $x$ and $v$ and let $p_{vz}$ be a
  shortest path between $v$ and $z$. 
  Then (by the triangle inequality? \XXX)
  \[
  |p_{xv}|+|p_{vz}|=|p_{vx}|+|p_{vz}|\ge\Delta_G\enspace.
  \]
  It is easy to see that $|p_{vu}|+|p_{vw}|\ge|p_{vx}|+|p_{vz}|$ because the
  vertices $u$ and $w$ are chosen so that to maximize this sum. Hence
  \[
  \tilde\Delta_G = |p_{vu}|+|p_{vw}| \ge\Delta_G\enspace.%\qedhere
  \]
\end{proof}

The impact of using $\tilde\Delta_G$ when computing the sample size $k$ is that
the sample will contain at most $c/\varepsilon^2$ more paths than if we used the
exact value $\Delta_G$. The computation of $\tilde\Delta_G$ has no (\XXX or
minimal) impact on the running time of our algorithm: once the $\Delta_G$ has
been computed, we can sample another node $u\neq v$ and then sample one of the
(already computed) shortest paths between $v$ and $v$ and use this path as first
element of the sample.

\subsection{Discussion}\label{sec:discussion}
\XXX TBD: running time, worst case, comparison with best previous work.

\citet{BrandesP07} present an algorithm that also uses sampling to approximate
the betweenness centrality of all the vertices of the graph. The algorithm
create a sample $S=\{v_1,\dotsc,v_k\}$ of $k$ vertices drawn uniformly at random 
and computes all the shortest paths between each $v_i$ to all other vertices in
the graph. The estimation $\tilde\betw'(u)$ for the betweenness centrality
$\betw(u)$ is
\[ 
\tilde\betw'(u)= \frac{n}{k}\sum_{v_i\in S}\sum_{w\neq v_i,w\neq
u}\sum_{p\in\mathcal{S}_{v_iw}}\frac{\mathds{1}_{\mathsf{Int}(p)}(u)}{|\mathcal{S}_{v_iw}|},
\]
As it was for our algorithm, the key ingredient to ensure a correct
approximation for the betweenness centrality is the computation of the sample
size $k$. Inspired by the work of~\citet{EppsteinW04}, \citet{BrandesP07} rely
on the following classical result by~\citet{Hoeffding63} for their computation
of $k$.

\begin{theorem}[\citep{Hoeffding63}]
  Let $X_1,X_2,\dotsc,X_k$ be a sequence of independent random variables
  such that $a_i\leq X_i\leq b_i$. Then for any $\xi > 0$
  \begin{equation}\label{eq:hoeffding}
    \Pr\left(\frac{|X_1+\dotsb+X_k|}{k}-\mathbf{E}\left[\frac{|X_1+\dotsb+X_k|}{k}\right]|\geq
    \xi\right)\leq 2e^{-2k^2\xi^2/\sum_{i=1}^{k}(b_{i}-a_{i})^2}\enspace.
  \end{equation}
\end{theorem}

\citet{BrandesP07} use this result to compute a sample size $k$ sufficient to ensure that, for
a fixed vertex $u$,
\[ 
\Pr\left(|\betw(u)-\tilde\betw'(u)|>\varepsilon\binom{n}{2}\right)<\frac{\delta}{n}\enspace.
\]
In their setting $\xi=\varepsilon\binom{n}{2}$ and there is a variable $X_i$ for
each vertex $v_i$ in the sample:
\[ 
X_i=n\sum_{w\neq v_i,w\neq
u}\sum_{p\in\mathcal{S}_{v_iw}}\frac{\mathds{1}_{\mathsf{Int}(p)}(u)}{|\mathcal{S}_{v_iw}|}\enspace
.
\]
It is easy to see that $0\le X_i\le n(n-2)$. Moreover,
\begin{align*}
\mathbf{E}\left[\frac{|X_1+\dotsb+X_k|}{k}\right] &=
\frac{n}{k}\mathbf{E}\left[\sum_{v_i\in S}\sum_{w\neq v_i,w\neq
u}\sum_{p\in\mathcal{S}_{v_iw}}\frac{\mathds{1}_{\mathsf{Int}(p)}(u)}{|\mathcal{S}_{v_iw}|}\right]
=\frac{n}{k}\sum_{v\in V}\mathbf{E}\left[Y_v\sum_{w\neq v,w\neq
u}\sum_{p\in\mathcal{S}_{vw}}\frac{\mathds{1}_{\mathsf{Int}(p)}(u)}{|\mathcal{S}_{vw}|}\right] =
\\
&=\frac{n}{k}\sum_{v\in V}\mathbf{E}[Y_v]\sum_{w\neq v,w\neq
u}\sum_{p\in\mathcal{S}_{vw}}\frac{\mathds{1}_{\mathsf{Int}(p)}(u)}{|\mathcal{S}_{vw}|}
=\frac{n}{k}\sum_{v\in V}\frac{k}{n}\sum_{w\neq v,w\neq
u}\sum_{p\in\mathcal{S}_{vw}}\frac{\mathds{1}_{\mathsf{Int}(p)}(u)}{|\mathcal{S}_{vw}|}
= \betw(u),
\end{align*}
where $Y_v$ is a Bernoulli random variable taking value 1 if $v\in S$, and 0
otherwise, so $\Pr(Y_v=1)=k/n$.

Plugging these values in~\eqref{eq:hoeffding}, we have
\[
\Pr\left(|\betw(u)-\tilde\betw'(u)|>\varepsilon\binom{n}{2}\right)\leq
2e^{-2k^2\varepsilon^2\left(\frac{n(n-1)}{2}\right)^2/kn^2(n-2)^2}=2e^{-k\epsilon^2(n-1)^2/2(n-2)^2}\enspace.
\]
We want this probability to be at most $\delta/n$, so we need a sample of size
\[
k\geq \frac{2(n-2)^2}{\varepsilon^2(n-1)^2}\left(\ln n
+\ln\frac{1}{\delta} +\ln 2\right)\enspace.
\]
An application of the union bound over the $n$ vertices ensures a uniform guarantee on the quality of
all the estimation, with probability at least $1-\delta$.

