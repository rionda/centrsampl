\section{Algorithms}\label{sec:algo}
In this section we present our algorithms to compute a set of approximations for the
betweenness centrality of the (top-$K$) vertices in a graph through sampling,
with probabilistic guarantees on the quality of the approximations.

\subsection{Approximation for all the vertices}\label{sec:allvertapprox}
The intuition behind the algorithm to approximate the betweenness values of all
vertices is the following. Given a graph $G=(V,E)$
with vertex-diameter $\VD(G)$ and two parameters $\varepsilon,\delta\in(0,1)$
we first compute a sample size $r$ using~\eqref{eq:vceapprox} with
\[d=\lfloor\log_2(\VD(G)-2)\rfloor+1\enspace.\]
The resulting sample size is
\begin{equation}\label{eq:samplesize}
r=\frac{c}{\varepsilon^2}\left(\lfloor\log_2(\VD(G)-2)\rfloor+1+\ln\frac{1}{\delta}\right)\enspace.
\end{equation}
This is sufficient to achieve the desired accuracy
(expressed through $\varepsilon$) with the desired confidence (expressed through
$1-\delta$). The algorithm repeats the following steps $r$ times:
\emph{1.}~it samples a pair $u,v$ of distinct vertices uniformly at random,
\emph{2.}~it
computes the set $\mathcal{S}_{uv}$ of all shortest paths between $u$ and $v$,
\emph{3.}~it selects a path $p$ from $\mathcal{S}_{uv}$ uniformly at random,
\emph{4.}~it increases by $1/r$ the betweenness estimation of each vertex in
$\mathsf{Int}(p)$. Note that if the sampled vertices $u$ and $v$ are not
connected, we can skip steps 3 and 4 because we defined
$\mathcal{S}_{uv}=\{p_\emptyset\}$. Denoting with $S$ the set of the sampled
shortest paths, the \emph{unbiased} estimator $\tilde\betw(w)$ for the betweenness
$\betw(w)$ of a vertex $w$ is the sample average 
\[
\tilde\betw(w) = \frac{1}{r}\sum_{p\in S}
\mathds{1}_{\mathsf{Int}(p)}(w) = \frac{1}{r}\sum_{p\in S}
\mathds{1}_{\mathcal{T}_w}(p)\enspace.
\]

There are two crucial steps in this algorithm: the computation of $\VD(G)$ and
the sampling of a path uniformly at random from $\mathcal{S}_{uv}$. We first
deal with the latter, and then present a linear-time constant-factor approximation algorithm for $\VD(G)$.
%in Sect.~\ref{sec:diam}. 
Algorithm~\ref{alg:algorithm} presents the pseudocode of the algorithm,
including the steps to select a random path.  The
\texttt{computeAllShortestPaths(}$u,v$\texttt{)}  on
line~\ref{algline:shortestpaths} is a call to a modified Dijkstra's (or BFS)
algorithm to compute the set $\mathcal{S}_{uv}$, with the same modifications
as~\citep{Brandes01}. The \texttt{getDiameterApprox()} procedure computes an approximation for $\VD(G)$. % (see Sect.~\ref{sec:diam} for details).

\paragraph{Unique shortest paths} 
When, for each pair $(u,v)$ of vertices of $G$, either there is a unique
shortest path from $u$ to $v$ or $v$ is unreachable from $u$, 
%When each pair $(u,v)$ of vertices of $G$ has a unique shortest path connecting them or
%When there is an unique shortest
%path between each pair of vertices of $G$, 
 then one can apply Lemma~\ref{lem:vcdimuppboundunique} and obtain a smaller
sample size
\[
  r= \frac{c}{\varepsilon^2}\left(3+\ln\frac{1}{\delta}\right)
\]
to approximate the betweenness values of all the vertices. 
This is an interesting result: the number of samples needed to compute a good
approximation to all vertices is a \emph{constant} and completely
\emph{independent from $G$}. Intuitively, this means that the algorithm is
extremely fast on graphs with this property. Unique shortest paths are common or
even enforced in road networks by slightly perturbing the edge weights or having
a deterministic tie breaking policy~\citep{GeisbergerSS08}.

\ifproof
\else
\paragraph{Bounded-distance betweenness} For the case of
$k$-bounded-distance betweenness, the sample size on line~\ref{algline:forloop}
of Alg.~\ref{alg:algorithm} can be reduced to 
\[ 
  r= \frac{c}{\varepsilon^2}\left(\lfloor\log_2(k-1)\rfloor + 1 +\ln\frac{1}{\delta}\right)
\]
and the computation of the shortest paths on line~\ref{algline:shortestpaths}
can be stopped after we reached the vertices that are $k$ ``hops'' far from $u$.
\fi

\paragraph{Sampling a shortest path}
Our procedure to select a random shortest path from $\mathcal{S}_{uv}$ is
inspired by the dependencies accumulation procedure used in Brandes' exact
algorithm~\citep{Brandes01}. 
%Given a vertex $w$, ~\citet{Brandes01} showed how
%to compute, for every vertex $z$, the amount $\sigma_{wz}$ of shortest paths
%from $w$ to $z$, while computing the set $\mathcal{S}_{wz}$. 
Let $u$ and $v$ be the vertices sampled by our algorithm
(Step~\ref{algline:samplevertices} of Alg.~\ref{alg:algorithm}). We assume that $u$ and
$v$ are connected otherwise the only possibility is to select the empty path
$p_\emptyset$. Let $y$ be any vertex belonging to at least one shortest path
from $u$ to $v$. Following~\citet{Brandes01}, we can compute $\sigma_{uy}$ and
$\mathcal{S}_{uy}$ while we compute the set $\mathcal{S}_{uv}$ of all the
shortest paths from $u$ to $v$. We can then use this information to select a
shortest path $p$ uniformly at random from $\mathcal{S}_{uv}$ as follows. For each vertex
$w$ let $P_u(w)$ be the subset of neighbors of $w$ that are \emph{predecessors}
of $w$ along the shortest paths from $u$ to $w$. Let $p^*=\{v\}$. Starting from
$v$, we select one of its predecessors $z\in P_u(v)$ using weighted random
sampling: each $z\in P_u(v)$ has probability $\sigma_{uz}/\sum_{w\in
P_u(v)}\sigma_{uw}$ of being sampled. We add $z$ to $p^*$ and then repeat the
procedure for $z$. That is, we select one of $z$'s predecessors from $P_u(z)$
using weighted sampling and add it to $p^*$, and so on until we reach $u$. Note
that we can update the estimation of the betweenness of the internal vertices
along $p^*$ (the only ones for which the estimation is updated) as we compute
$p^*$.

\begin{lemma}\label{lem:samplpath}
  The path $p^*$ built according to the above procedure is selected uniformly at
  random among the paths in $\mathcal{S}_{uv}$.
\end{lemma}

\begin{proof}
%\begin{IEEEproof}
  The probability of sampling $p^*=(u,z_1,\dotsc,z_{|p^*|-2},v)$ equals to the
  product of the probabilities of sampling the vertices internal to $p^*$, hence
  \[
  \Pr(p^*)=\frac{\sigma_{uz_{|p^*|-2}}}{\sigma_{uv}}\frac{\sigma_{uz_{|p^*|-3}}}{\sigma_{uz_{|p^*|-2}}}\dotsb
  \frac{1}{\sigma_{uz_2}}=\frac{1}{\sigma_{uv}}
  \]
  where we used \citep[Lemma3]{Brandes01} which tells us that for $w\neq u$,
  \[
  \sigma_{uw}=\sum_{j\in P_u(w)}\sigma_{uj}
  \]
  and the fact that for $z_1$, which is a neighbor of $u$, $\sigma_{uz_1}=1$.
%\end{IEEEproof}
\end{proof}

\begin{algorithm}[ht]
  \SetKwInOut{Input}{Input}
  \SetKwInOut{Output}{Output}
  \SetKwComment{Comment}{//}{}
  \SetKwFunction{VertexDiameter}{getVertexDiameter}
  \SetKwFunction{SamplePair}{sampleUniformVertexPair}
  \SetKwFunction{SampleUniform}{sampleUniformPath}
  \SetKwFunction{PairShortestPaths}{computeAllShortestPaths}
   \DontPrintSemicolon
  %\dontprintsemicolon
  \Input{Graph $G=(V,E)$ with $|V|=n$, $\varepsilon,\delta\in(0,1)$}
  \Output{A set of approximations of the betweenness centrality of the vertices
  in $V$}
  \ForEach{$w\in V$}
  {
  $\tilde\betw(v)\leftarrow 0$
  }
  $\VD(G)\leftarrow$\VertexDiameter{G}\label{alg:diamcomp}\; 
  $r\leftarrow (c/\varepsilon^2)(\lfloor\log_2(\VD(G)-2)\rfloor+\ln(1/\delta))$\;
  \For{$i\leftarrow 1$ to $r$}
  {\label{algline:forloop}
  $(u,v)\leftarrow$\SamplePair{$V$}\label{algline:samplevertices}\;
  $\mathcal{S}_{uv}\leftarrow$\PairShortestPaths{$u,v$}\label{algline:shortestpaths}\;
  \If{$\mathcal{S}_{uv}\neq\{p_\emptyset\}$}
  {
  \Comment{Random path sampling and estimation update}
  $j\leftarrow v$\;
  $s\leftarrow v$\;
  $t\leftarrow v$\;
  \While{$t \neq u$} {
  sample $z\in P_s(t)$ with probability $\sigma_{uz}/\sigma_{us}$\;
  \If{$z\neq u$} {
  $\tilde\betw(z) \leftarrow \tilde\betw(z)+1/r$\;
  $s\leftarrow t$\;
  $t\leftarrow z$\;
  }
  }
  }
  } % end For
  \Return{$\{(v,\tilde\betw(v)), v\in V\}$}
  \caption{Computes approximations $\tilde\betw(v)$ of the betweenness
  centrality $\betw(v)$ for all vertices $v\in V$.}
  \label{alg:algorithm}
\end{algorithm}

\paragraph{Approximating the vertex-diameter}%\label{sec:diam}
The algorithm presented in the previous section requires the value of the
vertex-diameter $\VD(G)$ of the graph $G$ (line
\ref{alg:diamcomp} of Alg.~\ref{alg:algorithm}). 
Computing the exact value of $\VD(G)$ could be done by solving the All Pair
Shortest Paths (APSP) problem, and taking the shortest path with the maximum size.
Algorithms for exactly solving APSP problem such as Johnson's which runs in
$O(V^2\log V+VE)$ or Floyd-Warshall's ($\Theta(V^3)$), would defeat our
purposes: once we have all the shortest paths for the computation of
the diameter, we could as well compute the betweenness of all the vertices exactly. 
Given that Thm.~\ref{thm:eapprox} (and Thm.~\ref{thm:releapprox})  only
requires an upper bound to the VC-dimension of the range set, an approximation
of the vertex-diameter would be sufficient for our purposes. Several refined
algorithms for approximating the diameter are
known~\citep{AingwordCIM99,BoitmanisFL06,RodittyW12}, with various running times
and quality of approximations. We briefly present a well-known and simple
approximation algorithm that has the right balance of accuracy and speed
for our purposes.

Let $G=(V,E)$ be an \emph{undirected} graph where \emph{all the edge weights are
equal}.
It is a well-known result that one can obtain a $2$-approximation
$\widetilde\VD(G)$ of the vertex-diameter $\VD(G)$ of $G$ in time $O(V+E)$ in
the following way: 1.~select a vertex $v\in V$ uniformly at random, 2.~compute
the shortest paths from $v$ to all other vertices in $V$, and 3.~finally take
$\widetilde\VD(G)$ to be the sum of the lengths of the two shortest paths with
maximum size (which equals to the two longest shortest paths) from $v$ to two
distinct other nodes $u$ and $w$. 
\ifproof
The approximation guarantee follows from the next lemma.

\begin{lemma}\label{lem:diam}
  $\VD(G)\le\widetilde\VD(G)\le 2\VD(G)$.
\end{lemma}
\begin{proof}
%\begin{IEEEproof}
  Let $v\in V$ be a vertex that we choose uniformly at random from the set $V$.
  Let also $u,w\in V$ be the two vertices such that the sum of the sizes of the
  shortest paths $p_{vu}$ and $p_{vw}$ is maximized among all the shortest paths
  that have $v$ as a source.  We have $\widetilde\VD(G)\le 2\VD(G)$ because
  $|p_{vu}|,|p_{vw}|\le\VD(G)$, so $|p_{vu}|+|p_{vw}|\le 2\VD(G)$. To see
  that $\widetilde\VD(G)\ge\VD(G)$, consider a pair of vertices $x$ and $z$ such
  that the length of a shortest path between $x$ and $z$ is equal to $\VD(G)$.
  Let $p_{xv}$ be a shortest path between $x$ and $v$ and let $p_{vz}$ be a
  shortest path between $v$ and $z$. 
  From the properties of the shortest paths $p_{vu}$, $p_{vw}$ we have
  $|p_{vu}|+|p_{vw}|\geq |p_{vx}|+|p_{vz}|$. Since the graph is undirected
  $|p_{s,t}|=|p_{t,s}|$ for every $s,t\in V$.
  Therefore:
  \[
    \widetilde\VD(G) = |p_{vu}|+|p_{vw}|\geq |p_{vx}|+|p_{vz}| =
    |p_{xv}|+|p_{vz}|\ge\VD(G)\enspace. 
  \]
  For the last inequality we used the fact that since $\VD(G)$ is the size of
  the shortest path from $x$ to $z$, then every other path (in this case
  $p'_{xz}$ which is the merge of $p_{xv}$ and $p_{vz}$) has greater or equal
  length from $p_{x,z}$.
  %$\qed$
%\end{IEEEproof}
\end{proof}
\fi
In case we have multiple connected components in $G$, we compute an upper bound
to the vertex diameter of each component separately by running the above
algorithm on each component, and then taking the maximum. 
The connected components can be computed in $O(n+m)$ by traversing the graph in
a Breadth-First-Search (BFS) fashion starting from a random $v$.
%Let $\tilde\Delta_{G_{1}},\tilde\Delta_{G_{2}}, \ldots , \tilde\Delta_{G_{k}}$
%be the approximations of the diameter for each of the $k$ connected components
%of $G$ as if they were separate graphs.
%The output of the approximation for the graph $G$, is the maximum of the values
%$\tilde\Delta_{G_{1}},\tilde\Delta_{G_{2}}, \ldots , \tilde\Delta_{G_{k}}$.
The time complexity of the approximation algorithm in the case of multiple
connected components is again $O(n+m)$ since the sum of the vertices of
individual components is $n$ and the sum of edges is $m$. 

The use of the above $2$-approximation in the computation of the
sample size from line~\ref{algline:forloop} of Alg.~\ref{alg:algorithm} results
in at most $c/\varepsilon^2$ additional samples than if we
used the exact value $\VD(G)$. The computation of $\widetilde\VD(G)$
does not affect the running time of our algorithm: for the construction of
the first sample we can reuse the shortest paths from the sampled
vertex $v$ that we used to obtain the approximation. Specifically, we can sample
a new vertex $u\neq v$ and then choose with uniform probability one of the
(already computed) shortest paths between $v$ and $u$.

If the graph is directed and/or not all edge weights are equal, the computation
of a good approximation to $\VD(G)$ becomes more problematic. In particular,
notice that there is no relationship between $\VD(G)$ and $\mathsf{diam}(G)$
when $G$ is weighted, as the shortest path with maximum size may not be the
shortest path with maximum weight. In these cases, one can use the size (number
of vertices) of the largest Weakly Connected Component (WCC), as a loose upper
bound to $\VD(G)$. The WCC's can again be computed in $O(n+m)$ using BFS.  This
quantity can be as high as $n$ but for the computation of the sample size we use
its logarithm, mitigating the crudeness of the bound. In this case our sample
size is comparable to that proposed by~\citet{BrandesP07}. %when there is a
%single WCC, as it would depend on $\log n$,
Nevertheless the amount of work done per sample by our algorithm is still much
smaller (see Sect.~\ref{sec:discussion} and~\ref{sec:exper} for more
details). In practice, it is possible that the nature of the network suggests a
much better upper bound to the vertex-diameter of the graph, resulting in a
smaller sample size. %than what suggested by the worst case we just presented.  

\paragraph{Analysis}\label{sec:analysis}
%The algorithm we described and presented in 
Algorithm~\ref{alg:algorithm} offers
probabilistic guarantees on the quality of all approximations of the betweenness
centrality.
\begin{lemma}\label{lem:correctness}
  With probability at least $1-\delta$, all the approximations computed by the
  algorithm are within $\varepsilon$ from their real value:
  \[
  \Pr\left(\exists v\in V \mbox{ s.t. }
  |\betw(v)-\tilde\betw(v)|>\varepsilon\right)<\delta\enspace .
  \]
\end{lemma}

\begin{proof}
%\begin{IEEEproof}
  For each $p_{uv}\in\mathbb{S}_G$ let
  \[
  \pi_G(p_{uv})=\frac{1}{n(n-1)}\frac{1}{\sigma_{uv}}\enspace.
  \]
  It is easy to see that $\pi_G$ is a probability distribution and
  $\pi_G(p_{uv})$ is the probability of sampling the path $p_{uv}$ during an
  execution of the loop on line~\ref{algline:forloop} in
  Alg.~\ref{alg:algorithm}, given the way that the vertices $u$ and $v$ are
  selected and Lemma~\ref{lem:samplpath}.
  
  Consider the range set $\range_G$ and the probability distribution $\prob_G$.
  Let $S$ be the set of paths sampled during the execution of the algorithm.
  For $r$ as in~\eqref{eq:samplesize}, Thm.~\ref{thm:eapprox} tells us that the sample $S$ is a
  $\varepsilon$-approximation to $(\range_G,\prob_G)$ with probability at least
  $1-\delta$. Suppose that this is indeed the case, then from
  Def.~\ref{def:eapprox} and the definition of $\range_G$ we have that
  \[
  \left|\prob_G(\mathcal{T}_v) - \frac{1}{r}\sum_{p\in
  S}\mathds{1}_{\mathcal{T}_v}(p)\right|=\left|\prob_G(\mathcal{T}_v) -
  \tilde\betw(v)\right|\le\varepsilon, \forall v\in
  V\enspace.
  \]
  From the definition of $\prob_G$ we have
  \[
  \prob_G(\mathcal{T}_v)=\frac{1}{n(n-1)}\sum_{p_{uw}\in\mathcal{T}_v}\frac{1}{\sigma_{uw}}=\betw(v),
  \]
  which concludes the proof.
%\end{IEEEproof}
\end{proof}

\ifproof
\paragraph{Time and space complexity.} Clearly the runtime of the algorithm is
dominated by the computation of the shortest path at each step, which takes time
$O(|V|+|M|)$ if the graph is unweighted (BFS algorithm) and time
$O(|E|+|V|\log|V|)$ otherwise (Dijkstra's algorithm with Fibonacci heap).
This time must then be multiplied by $r$ as in~\eqref{eq:samplesize} to obtain
the final time complexity. The space requirements are dominated by the amount of
memory needed to store the graph, so they are either $O(|V|^2)$ if using an
adjacency matrix, or $O(|V|+|E|)$ if using $|V|$ adjacency lists.
\fi

\subsection{High-quality approximation of the top-$K$ betweenness
vertices}\label{sec:topk}
%\XXX It would be great to look at the paper by Ezra, ``Small-Size Relative
%$(p,\varepsilon)$-Approximations for Well-Behaved Range Spaces'', 2012,
%available on the arXiv. She presents a smaller bound to the sample size for a
%relative $(p,\varepsilon$-approximation if the range set is ``well-behaved''. We
%should check whether ours is. (Warning: the paper is not exactly an example of
%clarity\ldots)

Very often in practice one is interested only in identifying the vertices with
the highest betweenness centrality, as they are the ``primary actors'' in the
network. We present here an algorithm to compute a very high-quality
approximation of the set $\TOPK(K,G)$ of the top-$K$ betweenness vertices in a graph
$G=(V,E)$. Formally, let $v_1,\dotsc,v_n$ be a labelling of the vertices in $V$
such that $\betw(v_i)\ge\betw(v_j)$ for $1\le i<j\le n$. Then $\TOPK(K,G)$ is
defined as the set of vertices with betweenness at least $\betw(v_K)$:
\[
\TOPK(K,G)=\{(v,\betw(v)), ~:~ v\in V \mbox{ and } \betw(v)\ge\betw(v_K)\}\enspace.
\]
Note that $\TOPK(K,G)$ may contain more than $K$ vertices. 
%This is in line with
%the definition of the set of top-$k$ Frequent Itemsets in the market basket
%analysis setting \XXX add reference.

Our algorithm works in two phases. Each phase is basically a run of the
algorithm for approximating the betweenness of all vertices. %presented in the previous section. 
The two phases differ in the way they compute the number of paths to sample and
the additional operations at the end of each phase. In the first
phase, we compute a lower bound $\ell'$ to $\betw(v_K)$. In the second phase we
use $\ell'$ to compute the number of samples $r$ needed to obtain a relative
$(\ell',\varepsilon)$-approximation to $(\range_G,\pi_G)$. We use $r$ samples to approximate the betweenness of all vertices again, and
return a collection of vertices that is, with high probability, a superset of
$\TOPK(K,G)$.

Let $\widetilde\VD(G)$ be an upper bound to the vertex-diameter of $G$. Given
$\varepsilon,\delta\in(0,1)$, let $\delta',\delta''$ be two positive reals such
that $(1-\delta')(1-\delta'')\ge(1-\delta)$. Let
\[
r'=\frac{c}{\varepsilon^2}\left(\lfloor\log_2(\widetilde\VD(G)-2)\rfloor+1+\log\frac{1}{\delta'}\right)\enspace.
\]
Let $\tilde\betw'_k$ be the $K$-th highest estimated betweenness obtained using
Algorithm~\ref{alg:algorithm} where $r=r'$, and let
%\[
  $\ell'=\tilde\betw_K-\varepsilon$, %.
%\]
%Let
and
\[
r''=\frac{c'}{\varepsilon^2\ell'}\left((\lfloor\log_2(\widetilde\VD(G)-2)\rfloor+1)\log\frac{1}{\ell'}+\log\frac{1}{\delta''}\right)\enspace.
\]
We run Algorithm~\ref{alg:algorithm} with $r=r''$ and let $\tilde\betw''_K$ be
the so-obtained $K$-th highest estimated betweenness. Let
$\ell''=\min\{\tilde\betw''(v) / (1+\varepsilon) ~:~ v\in V \mbox{ s.t. }
\tilde\betw''(v)\ge\tilde\betw''_K\}$. We
return the collection $\widetilde\TOPK(K,G)$ of vertices $v$ such that
$\tilde\betw''(v)*(1+\varepsilon)/(1-\varepsilon)\ge\ell''$:
\[
\widetilde\TOPK(K,G)= \left\{v\in V ~:~
\tilde\betw''(v)\frac{1+\varepsilon}{1-\varepsilon}\ge\ell''\right\}\enspace.
\]
\ifproof
The pseudocode of the algorithm is presented in Algorithm~\ref{alg:topk}.

\begin{algorithm}[ht]
  \SetKwInOut{Input}{Input}
  \SetKwInOut{Output}{Output}
  \SetKwComment{Comment}{//}{}
  \SetKwFunction{VertexDiameter}{getVertexDiameter}
  \SetKwFunction{SamplePair}{sampleUniformVertexPair}
  \SetKwFunction{SampleUniform}{sampleUniformPath}
  \SetKwFunction{PairShortestPaths}{computeAllShortestPaths}
   \DontPrintSemicolon
  %\dontprintsemicolon
  \Input{a graph $G=(V,E)$ with $|V|=n$, a positive integer $K\le n$, real
  values $\varepsilon,\delta\in(0,1)$}
  \Output{a superset of $\TOPK(K,G)$, with high-quality estimation of the
  betweenness for the vertices in the returned set.}
  $\delta',\delta''\leftarrow$ two positive reals such that
  $(1-\delta')(1-\delta'')\ge(1-\delta)$\;
  $\widetilde\VD(G)\leftarrow$ upper bound to $\VD(G)$\;
  \Comment{First phase}
  $r'\leftarrow\frac{c}{\varepsilon^2}\left(\lfloor\log_2(\widetilde\VD(G)-2)\rfloor+1+\log\frac{1}{\delta''}\right)$\;
  $B'=\{(v,\tilde\betw'(v)) ~:~ v\in V\}\leftarrow$ output of Algorithm~\ref{alg:algorithm} with $r=r'$\;
  $\tilde\betw_K'\leftarrow$ $K$-th highest betweenness value from $B'$, ties
  broken arbitrarily\;
  $\ell'\leftarrow\tilde\betw_K'-\varepsilon$\;
  $r''\leftarrow\frac{c'}{\varepsilon^2\ell'}\left((\lfloor\log_2(\widetilde\VD(G)-2)\rfloor+1)\log\frac{1}{\ell'}+\log\frac{1}{\delta''}\right)$\;
  \Comment{Second phase}
  $B''=\{(v,\tilde\betw''(v)) ~:~ v\in V\}\leftarrow$ output of
  Algorithm~\ref{alg:algorithm} with $r=r''$\;
  $\tilde\betw_K''\leftarrow$ $K$-th highest betweenness value from $B''$, ties
  broken arbitrarily\;
  $\ell''\leftarrow\min\{\tilde\betw''(v)/(1+\varepsilon) ~:~ v \mbox{ s.t. }
  \tilde\betw''(v)\ge\tilde\betw_K''\}$\;
  \Return $\{(v,\tilde\betw''(v)) ~:~ v\in V \mbox{ s.t. }
  \tilde\betw''(v)\frac{1+\varepsilon}{1-\varepsilon}\ge\ell''\}$\;
  \caption{High-quality approximation of the top-$K$ betweenness vertices}
  \label{alg:topk}
\end{algorithm}

\paragraph{Analysis}
The following lemma shows the properties of the collection $\widetilde\TOPK(K,G)$.
\fi

\begin{lemma}
  With probability at least $1-\delta$, 
  \begin{enumerate*}
    \item $\TOPK(K,G)\subseteq \widetilde\TOPK(K,G)$, and
    \item for all $v\in\TOPK(K,G)$ we have
      $|\tilde\betw''(v)-\betw(v)|\le\varepsilon\betw(v)$, and
    \item no vertex $u\in\widetilde\TOPK(K,G)\setminus\TOPK(K,G)$ has an estimated
      betweenness greater than $\ell'(1+\varepsilon)$.
  \end{enumerate*}
\end{lemma}
\ifproof
\begin{proof}
%\begin{IEEEproof}
  We start by proving 1. From Thm.~\ref{thm:eapprox} we know that, with probability at least
  $1-\delta'$, a sample of size $r'$ is a $\varepsilon$-approximation to
  $(\range_G,\pi_G)$ and from Thm.~\ref{thm:releapprox} we have that with
  probability at least $1-\delta''$ a sample of size $r''$ is a relative
  $(\ell',\varepsilon)$-approximation to $(\range_G,\pi_G)$. Suppose both these
  events occur, which happens with probability at least $1-\delta$. Then it is
  easy to see that $\ell'\le\betw(v_K)$, as there must be at least $K$ vertices
  with exact betweenness greater or equal to $\ell'$.  Consider now $\ell''$.
  Following the same reasoning as for $\ell'$, it should be clear that
  $\ell''\le\betw(v_K)$. The vertices included in $\widetilde\TOPK(K,G)$ are all and
  only those vertices that \emph{may} have exact betweenness at least $\ell''$,
  which implies that all vertices that have exact betweenness at least
  $\betw(v_K)$ are included in $\widetilde\TOPK(K,G)$. 
  Points 2 and 3 in the thesis follow from the properties of the relative
  $(\ell',\varepsilon)$-approximation (Def.~\ref{def:releapprox}).
%\end{IEEEproof}
\end{proof}
\else
We refer the reader interested in the proof to the extended online version of
the paper~\citep{RiondatoK13}.
\fi

The advantage of using our algorithm to approximate the collection of top-$K$
betweenness vertices consists in the very high-quality approximation of the
betweenness values for the returned set of vertices: they are all within a
multiplicative factor $\varepsilon$ from their exact values. By reverse-sorting
them according to the approximated betweenness, one can obtain a ranking that is
very similar to the original exact one. Previous algorithms could not achieve such a
good ranking as they were only able to approximate the betweenness values to
within an additive error $\varepsilon$. The cost of computing the high quality
approximation for the top-$K$ vertices is the cost of an additional run of our
algorithm to compute good approximations for all the vertices.

\subsection{Discussion}\label{sec:discussion}
\citet{JacobKLPT05} and independently~\citet{BrandesP07} present a
sampling-based algorithm to approximate the betweenness centrality of all the
vertices of the graph. The algorithm (which we call \textsf{BP}) creates a
sample $S=\{v_1,\dotsc,v_r\}$ of $r$ vertices drawn uniformly at random and
computes all the shortest paths between each $v_i$ to all other vertices in the
graph. Their estimation $\tilde\betw_\mathsf{BP}(u)$ for $\betw(u)$ is
\[ 
\tilde\betw_\mathsf{BP}(u)= \frac{1}{(n-1)r}\sum_{v_i\in S}\sum_{\substack{w\neq
v_i\\w\neq
u}}\sum_{p\in\mathcal{S}_{v_iw}}\frac{\mathds{1}_{\mathsf{Int}(p)}(u)}{|\mathcal{S}_{v_iw}|}\enspace.
\]
As it was for Algorithm~\ref{alg:algorithm}, the key ingredient to ensure a correct
approximation for the betweenness centrality is the computation of the sample
size $r$. Inspired by the work of~\citet{EppsteinW04}, \citet{BrandesP07} 
%rely
%on the following classical result by~\citet{Hoeffding63} for their computation
%of $r$.
%\begin{theorem}[\citep{Hoeffding63}]
%  Let $X_1,X_2,\dotsc,X_r$ be a sequence of independent random variables
%  such that $a_i\leq X_i\leq b_i$. Then for any $\xi > 0$
%  \begin{align}\label{eq:hoeffding}
%    \begin{split}
%      \Pr&\left(\left|\frac{|\sum_{i=1}^rX_i|}{r}-\mathbf{E}\left[\frac{|\sum_{i=1}^rX_i|}{r}\right]\right|\geq
%    \xi\right)\le \\
%      &\leq 2e^{-2r^2\xi^2/\sum_{i=1}^{r}(b_{i}-a_{i})^2}\enspace.
%    \end{split}
%  \end{align}
%\end{theorem}
%
%\citet{BrandesP07} use this result to compute a sample size $r$ sufficient to ensure that, for
%a fixed vertex $u$,
%\[ 
%\Pr\left(|\betw(u)-\tilde\betw'(u)|>\varepsilon\right)<\frac{\delta}{n}\enspace.
%\]
%In their setting $\xi=\varepsilon$ and there is a variable $X_i$ for
%each vertex $v_i$ in the sample:
%\[ 
%X_i=\frac{1}{n-1}\sum_{w\neq v_i,w\neq
%u}\sum_{p\in\mathcal{S}_{v_iw}}\frac{\mathds{1}_{\mathsf{Int}(p)}(u)}{|\mathcal{S}_{v_iw}|}\enspace
%.
%\]
%It is easy to see that $0\le X_i\le 1$. Moreover,
%\[
%  \mathbf{E}\left[\frac{|\sum_{i=1}^r X_i}{r}\right] = \betw(u)\enspace.
%\]
%%\begin{align*}
%%\mathbf{E}\left[\frac{|X_1+\dotsb+X_r|}{r}\right] &=
%%\frac{1}{r(n-1)}\mathbf{E}\left[\sum_{v_i\in S}\sum_{w\neq v_i,w\neq
%%u}\sum_{p\in\mathcal{S}_{v_iw}}\frac{\mathds{1}_{\mathsf{Int}(p)}(u)}{|\mathcal{S}_{v_iw}|}\right]
%%=\frac{1}{r(n-1)}\sum_{v\in V}\mathbf{E}\left[Y_v\sum_{w\neq v,w\neq
%%u}\sum_{p\in\mathcal{S}_{vw}}\frac{\mathds{1}_{\mathsf{Int}(p)}(u)}{|\mathcal{S}_{vw}|}\right] =
%%\\
%%&=\frac{1}{r(n-1)}\sum_{v\in V}\mathbf{E}[Y_v]\sum_{w\neq v,w\neq
%%u}\sum_{p\in\mathcal{S}_{vw}}\frac{\mathds{1}_{\mathsf{Int}(p)}(u)}{|\mathcal{S}_{vw}|}
%%=\frac{1}{r(n-1)}\sum_{v\in V}\frac{r}{n}\sum_{w\neq v,w\neq
%%u}\sum_{p\in\mathcal{S}_{vw}}\frac{\mathds{1}_{\mathsf{Int}(p)}(u)}{|\mathcal{S}_{vw}|}
%%= \betw(u),
%%\end{align*}
%%where $Y_v$ is a Bernoulli random variable taking value 1 if $v\in S$, and 0
%%otherwise, so $\mathbf{E}[Y_v]=\Pr(Y_v=1)=r/n$.
%Plugging these values in~\eqref{eq:hoeffding}, we have
%\[
%\Pr\left(|\betw(u)-\tilde\betw'(u)|>\varepsilon\right)\leq
%2e^{-2r\epsilon^2}\enspace.
%\]
%We want this probability to be at most $\delta/n$, so we need a sample of size
%They
prove that, to obtain good (within $\varepsilon$) estimations for the
betweenness of all vertices with probability at least $1-\delta$, it must be
\[
r\geq \frac{1}{2\varepsilon^2}\left(\ln n + \ln 2 +\ln\frac{1}{\delta}\right)\enspace.
\]
%An application of the union bound over the $n$ vertices ensures a uniform
%(simultaneous) guarantee on the quality of
%all the estimation, with probability at least $1-\delta$.
From this expression it should be clear that this sample size is usually much
larger than ours, as in practice $\VD(G)\ll n$. For the same reason, this
algorithm would not scale well as the network size increases (see also
Sect.~\ref{sec:exper}).

Another interesting aspect in which our algorithm and \textsf{BP} differ is the
\emph{amount of work done per sample}. Our algorithm computes a single set
$\mathcal{S}_{uv}$ for the sampled pair of vertices $(u,v)$: it performs a run
of Dijkstra's algorithm (or of BFS) from $u$, stopping when $v$ is reached.
\textsf{BP} instead computes all the sets $\mathcal{S}_{uw}$ from the sampled
vertices $u$ to all other vertices, again with a single run of Dijkstra or BFS,
but without the ``early-stopping condition'' that we have when we reach $v$.
Although in the worst case the two computations have the same time
complexity\footnote{It is a well-known open problem whether there is an
algorithm to perform a single $s,t$-shortest path computation between a pair of
vertices with smaller worst-case time complexity than the Single Source Shortest
Path computation.}, in practice we perform many fewer operations, as we can
expect $v$ not to always be very far from $u$ and therefore we can terminate
early. This fact has a huge impact on the running time. Our algorithm also
touches \emph{many fewer} edges than \textsf{BP}. The latter can touch all the
edges in the graph \emph{at every sample}, while our computation exhibits a much
higher \emph{locality}, exploring only a neighborhood of $u$ until $v$ is
reached. The results of our experimental evaluation presented in
Sect.~\ref{sec:exper} highlights this and other advantages of our method over
the one from~\citep{JacobKLPT05,BrandesP07}. In the future, we plan to
investigate the possibility of using \emph{bidirectional A$^*$
search}~\citep{Pohl69,KaindlK97} to further speed up the computation for each
sample of our algorithms.

\ifproof
The analysis of Algorithm 1 allow us to obtain a tighter analysis for the
algorithm by~\citet{BrandesP07} and~\citet{JacobKLPT05}.

\begin{lemma}\label{lem:variance}
  Fix $r>0$ and let $w\in V$. Let $\tilde\betw_\mathsf{BP}(w)$ be the estimation
  of the betweenness $\betw(w)$ computed by \textsf{BP} using $r$ samples, and
  let $\tilde\betw(w)$ be the estimation computed by
  Algorithm~\ref{alg:algorithm} using $r$ samples. The estimator
  $\tilde\betw_\mathsf{BP}(w)$ is \emph{uniformly better} than the estimator
  $\tilde\betw(w)$. I.e., for any value of $\betw(w)$ and $r$, we have
  \[
  \var[\tilde\betw_\mathsf{BP}(w)]\le\var[\tilde\betw(w)]\enspace.
  \]
\end{lemma}

\begin{proof}
  Consider the quantity
  \[
  \var[\tilde\betw(w)]-\var[\tilde\betw_\mathsf{BP}(w)]\enspace.\]
  We will show that the above quantity is greater than or equal to 0, proving
  the thesis.
  Since
  $\exp[\tilde\betw(w)]=\exp[\tilde\betw_\mathsf{BF}(w)]=\betw(w)$
  and using the definition of variance, we only need to show 
  \begin{equation}\label{eq:inequal}
    \exp[(\tilde\betw(w))^2]-\exp[(\tilde\betw_\mathsf{BF}(w))^2]\ge
    0\enspace.
  \end{equation}

  Let start from computing $\exp[(\tilde\betw_\mathsf{BF}(w))^2]$. For every
  vertex $v$, we define $\alpha_v$ as
  \[
  \alpha_v=\frac{1}{n-1}\sum_{\substack{u\in V
  \\u\neq
  v}}\sum_{p\in\mathcal{S}_{vu}}\frac{\mathds{1}_{\mathsf{Int}(p)}(w)}{\sigma_{vu}}\enspace.
  \]
  Note that 
  \begin{equation}\label{eq:betwalpha}
    \betw(w)=\frac{1}{n}\sum_{v\in V}\alpha(v)\enspace.
  \end{equation}
  Consider, for $1\le i \le r$, the contribution $X_i$ to $\tilde\betw_\mathsf{BP}(w)$
  of the paths computed from the $i$\textsuperscript{th} sampled vertex. $X_i$
  is a random variable that takes value $\alpha_v$ with probability $1/n$, for
  all $v\in V$. We have
  \begin{equation}\label{eq:xiexpvar}
    \exp[X_i]=\sum_{v\in V}\frac{1}{n}\alpha_v=\betw(w) \mbox{ and }
    \exp[X_i^2]=\frac{1}{n}\sum_{v\in V}\alpha_v^2, \forall 1\le i\le r\enspace.
  \end{equation}
  Clearly
  \[
  \tilde\betw_\mathsf{BP}(w)=\frac{1}{r}\sum_{i=1}^r X_i\enspace.
  \]
  The variables $X_i$'s are independent and identically distributed so
  \begin{align}\label{eq:expxisquared}
    \exp[(\tilde\betw_\mathsf{BP}(w))^2]&=\frac{1}{r^2}\exp\left[\left(\sum_{i=1}^rX_i\right)^2\right]=\frac{1}{r^2}\sum_{i=1}^r\left(\exp[X_i^2]+\sum_{j=i+1}^r(\exp[X_i]\exp[X_j])\right)=\frac{1}{r}\frac{1}{n}\sum_{v\in
    V}\alpha_v^2+\frac{r-1}{r}(\betw(w))^2,
\end{align}
where we used~\eqref{eq:xiexpvar}.

We now compute $\exp[(\tilde\betw(w))^2]$. 
Consider the contribution $Y_i$ to $\tilde\betw(w)$ of the
$i$\textsuperscript{th} sampled path by Algorithm 1, for $1\le i\le r$. $Y_i$ is a random variable that takes value
$\mathds{1}_{\mathsf{Int}(p)}(w)$ with probability $\pi(p)$, for every shortest
path $p\in\mathbb{S}_G$. We have
\begin{equation}\label{eq:yiexpvar}
  \exp[Y_i]=\exp[Y_i^2]=\betw(w), \forall 1\le i \le r\enspace.
\end{equation}
By definition, 
\[
\tilde\betw(w)=\frac{1}{r}\sum_{i=1}^rY_i\enspace.
\]
The random variables $Y_i$ are independent and identically distributed so
\begin{align}\label{eq:expyisquared}
  \exp[(\tilde\betw(w))^2]&=\frac{1}{r^2}\exp\left[\left(\sum_{i=1}^r
  Y_i\right)^2\right]=\frac{1}{r^2}\sum_{i=1}^r\left(\exp[Y_i^2]+\sum_{j=i+1}^r\exp[Y_i]\exp[Y_j]\right)=\frac{1}{r}\betw(w)+\frac{r-1}{r}(\betw(w))^2,
\end{align}
where we used~\eqref{eq:yiexpvar}.

We can now rewrite the left side of~\eqref{eq:inequal} using
~\eqref{eq:betwalpha},~\eqref{eq:expxisquared},~and~\eqref{eq:expyisquared}:
\begin{align*}
  \exp[(\tilde\betw(w))^2]-\exp[(\tilde\betw_\mathsf{BF}(w))^2] &=
  \frac{1}{r}\betw(w)+\frac{r-1}{r}(\betw(w))^2 - \frac{1}{r}\frac{1}{n}\sum_{v\in
    V}\alpha_v^2
   -\frac{r-1}{r}(\betw(w))^2 =
    \frac{1}{r}\frac{1}{n}\sum_{v\in V}(\alpha_v - \alpha_v^2)\enspace. 
\end{align*}
Since $\alpha_v\in[0,1]$, we have $\alpha_v-\alpha_v^2\ge 0$ for all $v$, and our proof
is complete.
\end{proof}

The following lemma is an easy consequence of the above.

\begin{lemma}\label{lem:MSE}
  \textsf{BP} has lower expected Mean Squared Error than Algorithm~1:
  \[
  \exp[\mse_\mathsf{BP}]\le \exp[\mse]\enspace.
  \]
\end{lemma}

\begin{proof}
  We have
  \[ 
  \exp[\mse_\mathsf{BP}] = \exp\left[\frac{1}{n}\sum_{v\in
  V}\left(\tilde\betw_\mathsf{BP}(w)-\betw(v)\right)^2\right]=\frac{1}{n}\sum_{v\in
  V}\exp\left[\left(\tilde\betw_\mathsf{BP}(w)-\betw(v)\right)^2\right]=\frac{1}{n}\sum_{v\in
  V}\var[\tilde\betw_\mathsf{BP}(w)],
  \]
  where we used the linearity of expectation and the fact that the estimator
  $\tilde\betw_\mathsf{BP}(w)$ is unbiased for $\betw(w)$ ($\exp[\tilde\betw_\mathsf{BP}(w)]=\betw(w)]$).
  Analogously for Algorithm~1:
  \[
  \exp[\mse] = \frac{1}{n}\sum_{v\in V}\var[\tilde\betw(w)]\enspace.
  \]
  Hence
  \[
  \exp[\mse]-\exp[\mse_\mathsf{BP}]=\frac{1}{n}\sum_{v\in
  V}\left(\var[\tilde\betw(w)]-\var[\tilde\betw_\mathsf{BP}(w)]\right),
  \]
  and from Lemma~\ref{lem:variance} we have that each addend of the sum is
  non-negative and so is the above expression, concluding our proof.
\end{proof}

%\begin{lemma}
%  Let $G=(V,E)$ be a graph, and $\varepsilon,\delta\in(0,1)$. Let
%  \[
%  r =
%  \frac{c}{\varepsilon^2}\left(\lfloor\log_2(\mathsf{VD}(G)-2\rfloor+1+\ln\frac{1}{\delta}\right)\enspace.
%  \]
%  and let $S$ be a set of vertices of $G$ sampled uniformly at random with
%  replacement from $V$. Let $\tilde\betw_{\mathsf{BP}}(w)$ be the estimated
%  value for $\betw(w)$ computed by \textsf{BP} using $S$. Then
%  \[
%  \Pr(\exists w\in V \mbox{ s.t. }
%  |\tilde\betw_{\mathsf{BP}}(w)-\betw(w)|>\varepsilon)<\delta\enspace.
%  \]
%  %If Algorithm 1 computes an $(\varepsilon,\delta)$-approximation with $r$
%  %samples, then the algorithm by~\citet{BrandesP07} does the same with the same
%  %number of samples.
%\end{lemma}
%
%\begin{proof}
%  Consider a simple coupling of the executions of Algorithm 1 and \textsf{BP}
%  for $r$ samples: when Algorithm 1 samples a shortest path from a vertex $v$ to a vertex $u$,
%  \textsf{BP} samples the vertex $v$. 
%
%Consider an execution of the algorithms and let $X_v$ be the number of times
%that vertex $v$ is the starting point of a sampled path. 
%
%The estimation for the betweenness of vertex $w$ computed by \textsf{BP} is
%\[
%\tilde\betw_{\mathsf{BP}}(w)=\frac{1}{r}\sum_{v\in V}
%X_v\underbrace{\left(\frac{1}{n-1}\sum_{\substack{u\in V \\u\neq
%v}}\sum_{p\in\mathcal{S}_{vu}}\frac{\mathds{1}_{\mathsf{Int}(p)}(w)}{|\mathcal{S}_{vu}|}\right)}_{Y_v(w)}\enspace.
%\]
%The estimation for the betweenness of vertex $w$ computed by Algorithm 1
%is
%\[
%\tilde\betw_{\mathrm{A1}}(w)=\frac{1}{r}Z_w,
%\]
%where $Z_w$ is the number of times that a path containing the vertex $w$ gets
%sampled. Now, for each $v$ and for $1\le i\le X_v$, let $p_v^{(i)}$ be the path
%sampled the $i$\textsuperscript{th} time that the sampled path had $v$ as
%starting vertex, and let $A_v^{(i)}(w)=1$ if $w\in\mathsf{Int}(p_v^{(i)}$, and
%$A_v^{(i)}(w)=0$ otherwise. Then it is clear that 
%\[
%\tilde\betw_{\mathrm{A1}}(w)=\frac{1}{r}\sum_{v\in V}
%X_v\frac{\sum_{i=1}^{X_v}A^{(i)}_v(w)}{X_v}\enspace.
%\]
%We claim that
%\[
%\mathbb{E}\left[\left.\frac{\sum_{i=1}^{X_v}A^{(i)}_v(w)}{X_v}\right|X_v\right]=Y_v(w),\]
%where the expectation is taken over the choice of the paths, which are sampled
%according to the distribution $\pi$.
%To see this, consider $\mathbb{E}[A_v^{(i)}(w) ~|~ X_v]$. It
%is clear that $A^{(i)}(w)$ and $X_v$ are independent, for $1\le i\le X_v$. From
%this and the definition of expectation we have 
%\[
%\mathbb{E}\left[\left.A_v^{(i)}(w)\right|X_v\right] =
%\mathbb{E}\left[A_v^{(i)}(w)\right]=\frac{1}{n-1}\sum_{\substack{u\in V
%\\u\neq
%v}}\sum_{p\in\mathcal{S}_{vu}}\frac{\mathds{1}_{\mathsf{Int}(p)}(w)}{|\mathcal{S}_{vu}|}
%=Y_v(w)\enspace.
%\]
%This expression does not depend on $i$.
%
%The claim then follows easily from the linearity of expectation:
%\[
%\mathbb{E}\left[\left.\frac{\sum_{i=1}^{X_v}A^{(i)}_v(w)}{X_v}\right|X_v\right] =
%\sum_{i=1}^{X_v}\mathbb{E}\left[\left.\frac{A^{(i)}_v(w)}{X_v}\right|X_v\right]=\sum_{i=1}^{X_v}\frac{1}{X_v}\mathbb{E}\left[A_v^{(i)}(w)\right]=\mathbb{E}\left[A_v^{(1)}(w)\right]=Y_v(w)\enspace.
%\]
%Recall from Lemma~\ref{lem:variance} that the estimator
%$\tilde\betw_{\mathsf{BP}}(w)$ has lower variance than the estimator
%$\tilde\betw(w)$, for all vertices $w$. 
%
%\XXX: The following is not actually true \MR. Since both estimators have
%the same expectation $\betw(w)$, this means that for any $a\in[0,1]$ we have
%\[
%\Pr\left(|\tilde\betw_{\mathsf{BP}}(w)-\betw(w)|>a\right)\le\Pr(\left(|\tilde\betw(w)-\betw(w)|>a\right)\enspace.
%\]
%
%This means that for every collection of $r$ samples for which Algorithm 1 computes estimates
%$\tilde\betw(w)$ such that
%\begin{equation}\label{eq:edapprox}
%|\tilde\betw(w)-\betw(w)|\le\varepsilon, \forall w\in W,
%\end{equation}
%so would \textsf{BP} using the $r$ samples according to the coupling. For
%Algorithm $1$ the event in~\eqref{eq:edapprox} occurs with probability at least
%$1-\delta$ over the choice of samples, and so would be for \textsf{BP},
%concluding our proof.
%\end{proof}
%

The estimator $\tilde\betw_{\mathsf{BP}}(w)$ has lower variance and MSE than
$\tilde\betw(w)$, which means that it can give better estimations of the
betweenness values in practice using the same number of samples. Before always
opting for \textsf{BP} (or for the algorithm by~\citet{GeisbergerSS08}, whose
estimators have even lower variance than those of \textsf{BP}) with a number of
samples equal to the one in~\eqref{eq:samplesize}, one should
nevertheless take into account two facts. Firstly, we do not currently have a
proof that \textsf{BP} (or the algorithm from~\citep{GeisbergerSS08}) can
compute, with a number of samples as in~\eqref{eq:samplesize},
a good (within $\pm\varepsilon$) approximation of the betweenness of all
vertices with probability at least $1-\delta$. We conjecture this fact could be
proven using pseudodimension~\citep[Chap.~11]{AnthonyB99}. Secondly we already
argued that per sample, the computation of $\tilde\betw_{\mathsf{BP}}(w)$
requires more time than the one for $\tilde\betw_{\mathrm{A1}}(w)$. The
difference is even larger if Algorithm~\ref{alg:algorithm} uses bidirectional
search~\citep{KaindlK97,Pohl69}. 

We can conclude this discussion stating that Algorithm~\ref{alg:algorithm},
\textsf{BP}, and the algorithm by~\citet{GeisbergerSS08} share the same design
principles, but choose different trade-offs between accuracy and speed.  
\fi

