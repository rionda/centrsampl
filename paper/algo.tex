\section{Algorithm}\label{sec:algo}
In this section we present our algorithm to compute a set of approximations for the
betweenness centrality of all vertices in a graph through sampling, with
probabilistic guarantees on the quality of the approximations.

The intuition behind the algorithm is the following. Given a graph $G=(V,E)$
with vertex-diameter $\Delta_g$ and two parameters $\varepsilon,\delta\in(0,1)$ we first compute a sample
size $k$ using~\eqref{eq:vceapprox} with
$d=\lfloor\log_2\Delta_G\rfloor+1$:
\begin{equation}\label{eq:samplesize}
k=\frac{c}{\varepsilon^2}\left(\lfloor\log_2\Delta_G\rfloor+1+\ln\frac{1}{\delta}\right)\enspace.
\end{equation}
The sample size is sufficient to ensure
to achieve the desired accuracy (expressed through $\varepsilon$) with the
desired confidence (expressed through $1-\delta$). Then the algorithm builds a
sample $S$ of $k$ shortest path from $\mathbb{S}_G$ by sampling them according to
the probability distribution $\prob_G$ defined on $\mathbb{S}_G$ as follows:
given a shortest path $p_{uv}\in\mathbb{S}_G$ between a pair of vertices
$(u,v)\in V\times V$, 
\[
\prob_G(p_{uv})=\frac{1}{\binom{n}{2}|\mathcal{S}_{uv}|}\enspace.%=\frac{2}{n(n-1)|\mathcal{S}_{uv}|}\enspace.
\]
Sampling according to $\prob_G$ can be achieved by first picking a pair of distinct vertices $(u,v)$
uniformly at random, computing the set $\mathcal{S}_{uv}$ of all the shortest
paths between $u$ and $v$, and selecting one of these shortest paths uniformly
at random. The estimation $\tilde\betw(v)$ of the betweenness centrality of each
vertex $v\in V$ is obtained by computing the fraction of shortest paths in $S$
that $v$ belongs to and de-normalizing it: 
\[
\tilde\betw(v) = \binom{n}{2}\frac{1}{k}\sum_{p\in S}
\mathds{1}_{p}(v) = \binom{n}{2}\frac{1}{k}\sum_{p\in S}
\mathds{1}_{\mathcal{T}_v}(p)\enspace.
\]
Algorithm \ref{alg:algorithm} presents the pseudocode.
\begin{algorithm}[ht]
  \SetKwInOut{Input}{Input}
  \SetKwInOut{Output}{Output}
  \SetKwFunction{VertexDiameter}{getVertexDiameter}
  \SetKwFunction{SamplePair}{sampleVertexPair}
  \SetKwFunction{SampleUniform}{sampleUniformPath}
  \SetKwFunction{PairShortestPaths}{computeAllShortestPaths}
   \DontPrintSemicolon
  %\dontprintsemicolon
  \Input{a graph $G=(V,E)$ with $|V|=n$, real values $\varepsilon,\delta\in(0,1)$}
  \Output{A set of approximations }
  $\Delta_G\leftarrow$\VertexDiameter{G}\label{alg:diamcomp}\; 
  $k\leftarrow (c/\varepsilon^2)(\lfloor\log_2\Delta_G\rfloor+\ln(1/\delta))$\;
  $S\leftarrow\emptyset$\;
  \For{$i\leftarrow 1$ to $k$} {
  $(u,v)\leftarrow$\SamplePair{$G$}\;
  $\mathcal{S}_{uv}\leftarrow$\PairShortestPaths{$(u,v)$}\;
  $s\leftarrow$\SampleUniform{$\mathcal{S}_{uv}$}\;
  $S\leftarrow S\cup\{s\}$\;
  }
  \For{$v\in V$} {
  $\tilde\betw(v)\leftarrow\binom{n}{2}\sum_{p\in S}\mathds{1}_{p}(v)$\;
  }
  \Return{$\{\tilde\betw(v), v\in V\}$}
  \caption{Computes approximations $\tilde\betw(v)$ of the betweenness
  centrality $\betw(v)$ for all vertices $v\in V$.}
  \label{alg:algorithm}
\end{algorithm}

The algorithm we describes offers probabilistic guarantees on the quality of all
approximations of the betweenness centrality.
\begin{lemma}\label{lem:correctness}
  With probability at least $1-\delta$, all the approximations computed by the
  algorithm are within $\varepsilon\binom{n}{2}$ from their real value:
  %The algorithm computes a set of approximations $\tilde\betw(v)$ for each $v\in
  %V$ such that
  \[
  \Pr\left(\exists v\in V \mbox{ s.t. }
  |\betw(v)-\tilde\betw(v)|>\varepsilon\binom{n}{2}\right)<\delta\enspace .
  \]
\end{lemma}

\begin{proof}
  Consider the range set $\range_G$ and the probability distribution $\prob_G$.
  For $k$ as in~\eqref{eq:samplesize}, the sample $S$ is a
  $\varepsilon$-approximation to $(\range_G,\prob_G)$ with probability at least
  $1-\delta$. Suppose that this is indeed the case, then from
  Def.~\ref{def:eapprox} and the definition of $\range_G$ we have that
  \[
  \left|\prob_G(\mathcal{T}_v) - \frac{1}{k}\sum_{p\in
  S}\mathds{1}_{\mathcal{T}_v}(p)\right|=\left|\prob_G(\mathcal{T}_v) -
  \frac{1}{\binom{n}{2}}\tilde\betw(v)\right|\le\varepsilon, \forall v\in
  V\enspace.
  \]
  From the definition of $\prob_G$ we have
  \[
  \prob_G(\mathcal{T}_v)=\frac{1}{\binom{n}{2}}\sum_{p_{uw}\in\mathcal{T}_v}\frac{1}{|\mathcal{S}_{uw}|}=\frac{1}{\binom{n}{2}}\betw(v),
  \]
  which concludes the proof.
\end{proof}

\subsection{Approximating the diameter}\label{sec:diam}
The algorithm presented in the previous section requires the knowledge (or the
computation) of the vertex-diameter $\Delta_G$) of the graph $G$ (line
\ref{alg:diamcomp} of Alg.~\ref{alg:algorithm}). Computing $\Delta_G$ exactly
can be done by finding the shortest paths between all pair of vertices in $V$
and computing the maximum size. This can be done in time \XXX. Such an exact
computation would defeat our purposes, because once we have all the shortest
paths, we can compute the betweenness of all the nodes exactly. Given that
Thm.~\ref{thm:eapprox} only requires an upper bound to the VC-dimension of the
range set, an upper bound to the vertex-diameter would be sufficient for our
purposes, provided its computation is fast and it does not result in a much
larger sample size than if we used the exact value for $\Delta_G$. There are
known algorithms the diameter of a graph \XXX references, discussion \MR
For our case, a $2$-approximation to $\Delta_G$ can be computed in time \XXX by selecting a
vertex $v\in V$ uniformly at random, computing the shortest path from $v$ to all
other vertices in $V$, and taking $\tilde\Delta_G$ to be the sum of the sizes of
the two longest shortest paths from $u$ to two distinct other nodes. 

\begin{lemma}
  $\tilde\Delta_G\le 2\Delta_G$.
\end{lemma}
\begin{proof}
  \XXX TBD but seems easy.
\end{proof}

The impact of using $\tilde\Delta_G$ when computing the sample size $k$ is that
the sample will contain at most $c/\varepsilon^2$ more paths than if we used the
exact value $\Delta_G$. 

\subsection{Discussion}
\XXX TBD: running time, worst case, comparison with best previous work.

