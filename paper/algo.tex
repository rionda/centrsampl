\section{Algorithm}\label{sec:algo}
In this section we present our algorithm to compute a set of approximations for the
betweenness centrality of all vertices in a graph through sampling, with
probabilistic guarantees on the quality of the approximations.

The intuition behind the algorithm is the following. Given a graph $G=(V,E)$ and
two parameters $\varepsilon,\delta\in(0,1)$ we first compute a \emph{sample
size} $k$ as function of $\varepsilon$, $\delta$, and the edge-diameter
$\Delta_G$, so that to ensure to achieve the desired accuracy (expressed through
$\varepsilon$) with the desired confidence (expressed through $1-\delta$). Then,
for $k$ iterations, the algorithm picks a pair of distinct vertices $(u,v)$
uniformly at random, computes all the shortest paths between $u$ and $v$,
and select one of these shortest paths uniformly at random. Let $\mathcal{S}$ be
the set of the so-obtained $k$ shortest paths. The estimation of the betweenness
centrality of each vertex $v\in V$ is obtained by counting the number of
shortest paths from $\mathcal{S}$ going through $v$ and de-normalizing this
count by multiplying it for $\binom{n}{2}/k$. Algorithm~\XXX %\ref{alg:algorithm}
presents the pseudocode.

\XXX TBD
