\section{Conclusions}\label{sec:concl}
In this work we presented two random-sampling-based algorithms for accurately and
efficiently estimate the betweenness centrality of the (top-$K$) vertices in a
graph, with high probability.
%For very large graphs, exact computation of the
%betweenness centrality is impossible, so one has to resort to an approximation.
Our algorithms are based on a novel application of VC-dimension theory, and
therefore take a different approach than previous ones achieving the same
guarantees~\citep{BrandesP07,GeisbergerSS08,JacobKLPT05}. The number of samples
needed to approximate the betweenness with the desired accuracy and confidence
does not depend on the number of vertices in the graph, but rather on a
characteristic quantity of the network that we call
\emph{vertex-diameter}. In some cases, the sample size is completely
independent from any property of the graph. %, which is interesting and unexpected. %
\ifproof
Our methods can be applied to many variants of betweenness, including edge
betweenness. %
\fi
Our algorithms perform much less work than previously presented methods. %offering
%the same approximation guarantee. 
As a consequence, they are much faster and
scalable, as verified in the extensive experimental
evaluation using many real and artificial graphs. 
\ifdmkd
\else
In future work we would like
to explore the possibility of using bidirectional A\textsuperscript{*}
search~\citep{Pohl69,KaindlK97} to further speed up our algorithms.  
\fi
%We are also
%interested in extending our methods to generalizations of betweenness
%centrality~\citep{KourtellisASIT12,DolevEP10} and to other centrality measures. 

\ifdmkd
\else
\paragraph*{Acknowledgements} This project was supported, in part, by the
National Science Foundation under award IIS-1247581. We are thankful to Eli
Upfal for his guidance and advice and to the anonymous reviewers of WSDM'14
whose comments helped us improving this work.
\fi

