\section{A range set for shortest paths}\label{sec:rangeset}
We now define a range set for shortest paths of a graph $G=(V,E)$, and present 
an upper bound to its VC-dimension. We show that this upper bound is strict. We
use the range set and the bound in our algorithm for estimating the betweenness
centrality of vertices of the graph.

The range set $\range_G$ is defined on the set $\mathbb{S}_G$ of all shortest
paths between vertices of $G$. It contains, for each vertex $v\in V$, the set
$\mathcal{T}_v$ of shortest paths that $v$ is internal to:
\[
\range_G = \{\mathcal{T}_v ~:~ v\in V\}\enspace.
\]

\begin{lemma}\label{lem:vcdimuppbound}
  Given a graph $G=(V,E)$ with vertex-diameter $\Delta_G$, the range set
  $\range_G$ associated to the shortest paths in $G$ has VC-dimension
  $\VC(\range_G)$ at most $\lfloor\log_2(\Delta_G-2)\rfloor+1$.
\end{lemma}

\begin{proof}
Let $k>\lfloor\log_2(\Delta_G-2)\rfloor+1$ and assume for the sake of contradiction
that $\VC(\range_G)=k$. From the definition of the VC-dimension there is a set
$Q\subseteq\mathbb{S}_G$ of size $k$ that is shattered by $\range_G$. Let $p$ be
an element of $Q$. The path $p$ is contained in $2^{k-1}$ non-empty subsets of
$Q$. Let us label this non-empty subsets of $Q$ containing $p$ as
$S_1,\dotsc,S_{2^{k-1}}$, where the labelling is arbitrary.
Given that $Q$ is shattered then, for each set $S_i$ there must be a range $R_i$ in
$\range_G$ such that $S_i=Q\cap R_i$. Since all the $S_i$'s are
different from each other, then all the $R_i$'s must be different from each
other. Given that $p$ belongs to each $S_i$, then $p$ must also belong to each
$R_i$, that is, there are $2^{k-1}$ distinct ranges in $\range_G$ containing
$p$. But $p$ belongs to the ranges corresponding to the internal vertices of
$p$, i.e., to the vertices in $\mathsf{Int}(p)$. This means that the number of ranges
in $\range_G$ that $p$ belongs to is equal to $|p|-2$. But $|p|\le\Delta_G$, by definition of $\Delta_G$, so $p$
can belong to at most $\Delta_G-2$ ranges from $\range_G$. Given that
$2^{k-1}>\Delta_G-2$, we reached a contradiction and there cannot be $2^{k-1}$
distinct ranges containing $p$, hence not all the sets $S_i$ can be expressed as
$Q\cap R_i$ for some $R_i\in\range_G$, but then $Q$ cannot be shattered and
$\VC(\range_G)\le\lfloor\log_2(\Delta_G-2)\rfloor+1$.%\qed
\end{proof}

\XXX TBD: Consider the case of local betweenness centrality. \MR

In the restricted case when the graph is undirected and every pair of distinct vertices
has a unique shortest path between them, the upper bound can be improved.
\begin{lemma}\label{lem:vcdimuppboundunique}
  Given an undirected graph $G=(V,E)$ such that $|\mathcal{S}_{uv}|\le1$ for all
  pairs $(u,v)\in V\times V$, the range set $\range_G$ associated to the
  shortest paths in $G$ has VC-Dimension $\VC(\range_G)$ at most $3$.
\end{lemma}

\begin{proof}
  First of all, notice that in this restricted setting, if two different paths
  $p_1$ and $p_2$ meet at a vertex $u$, then they either go on together or at a
  certain point they separate never to meet again at any other vertex $v=\neq u$.
  This is easy to see: if they could separate and then meet again, then there
  would be two distinct shortest paths between $u$ and $v$, which is a
  contradiction of the hypothesis. Let us denote this fact as $\mathsf{F}$.

  Assume now that $\VC(\range_G)>3$, then there must be a set
  $Q=\{p_1,p_2,p_3,p_4\}$ of four shortest paths that can be shattered by
  $\range_G$. Hence there must be a vertex $w$ such that $\mathcal{T}_{w}\cap
  Q=Q$. Note that $w$ can not be an end point for any of the paths in $Q$, by
  definition of $\mathcal{T}_w$. Hence all the paths in $Q$ must \emph{enter}
  and \emph{exit} $w$. We can assume to label some of the edges incident to $w$
  as ``entrance'' edges and the rest as ``exit'' edges, so that each path in $Q$
  enters $w$ through an entrance edge and exit through an exit edge. Given that
  the graph is undirected, we can always find such a labeling by appropriately
  reversing some of the paths in $Q$. In the following we will examine how the
  four paths ``enter'' and ``exit'' $w$, and show that all possible cases would
  result in a contradiction of $\mathsf{F}$. 

  \begin{description}
    \item[Case 1] The four paths in $Q$ enter $w$ through any combination of
      edges (e.g., three through the same edge and one from a different edge,
      two from the same edge and the other two from a single different edge, all four
      through four different edges, two from the same edge and the other two
      each from a different edge), except all four through the same edge. The four paths
      exit $w$ through any combination of edges, except all four through the
      same edge. Then it is always possible to find a pair $(p',p'')$ of paths
      in $Q$ such that $p'$ and $p''$ enter $w$ through different edges and exit
      $w$ through different edges. These two paths can not have met before
      entering $w$ as this would lead to a contradiction of $\mathsf{F}$ and
      analogously can not meet again after exiting $w$. Then it is not possible
      to find a vertex $z$ such that $\mathcal{T}_z\cap Q=\{p',p''\}$, which
      implies that $Q$ cannot be shattered by $\range_G$, contraddicting the
      hypothesis.
    %\item[Case 1] The four paths in $Q$ enter $w$ through four distinct edges
    %  and do not exit $w$ all through the same edge (we will analyze this
    %  situation in Case X). There there is a pair $(p',p'')$ of paths in $Q$ which exit $w$
    %  through two distinct edges. These two paths can not have met before
    %  entering $w$, as this would imply that they met, separate, and then met
    %  again in $w$, which would contraddict $\mathsf{F}$. Analogously, they can
    %  not meet after exiting $w$. Then there is no vertex $z$ such that
    %  $\mathcal{T}_z\cap Q=\{p',p''\}$, which implies that $Q$ cannot be
    %  shattered by $\range_G$, contraddicting the hypothesis.
    %\item[Case 2] 


    %\item[Case X-1] Three of the paths in $Q$ enter $w$ through the same edge
    %  and one through a different edge. Three paths exit $w$ through the same
    %  edge and one through a different edge. It is always possible to find a
    %  pair $(p',p'')$ such that $p'$ and $p''$ enter $w$ through different
    %  edges and exit $w$ through different edges. Then these two paths cannot
    %  have met before entering $w$ as this would lead to a contradiction of
    %  $\mathsf{F}$ and analogously can not meet again after exiting $w$. Then it
    %  is not possible to find a vertex $z$ such that $\mathcal{T}_z\cap
    %  Q=\{p',p''\}$, which implies that $Q$ cannot be shattered by $\range_G$,
    %  contraddicting the hypothesis.
    \item[Case 2] Suppose the four paths from $Q$ enter $w$ in any way
      (including all through the same edge) and all exit $w$ through the same
      edge. We can virtually contract all the edges belonging to all the four
      paths. It is easy to see that after this
      virtual contraction, there will be a (virtual) vertex $z$ that the paths
      enter (not all together) and exit (not all togheter). The way that they
      enter and exit $z$ falls in the previous case, for which we reached the  
      conclusion that $Q$ can not be shattered by $\range_G$, hence the same
      holds for this case. The case for when the four paths enter $w$ through
      the same edge and exit $w$ in any way can be obtained by simmetry by
      reversing all paths.
  \end{description}
\end{proof}

The bound presented in Lemma~\ref{lem:vcdimuppbound} is strict in the sense that
for each $d>2$ we can build a graph $G_d$ with vertex-diameter $d$ and such
that the range set associated to the set of shortest path of $G_d$ has
VC-dimension exactly $\lfloor\log_2(d-2)\rfloor+1$.

\begin{lemma}
  Given $d>2$ there exists a graph $G$ with vertex-diameter $d$ and such that
  $\VC(\range_G)=\lfloor\log_2(d-2)\rfloor+1$.
\end{lemma}

\begin{proof}
  \XXX TBD Remember to take care of the cases $d=1,2$.
\end{proof}

