\section{A range set for shortest paths}\label{sec:rangeset}
We now define a range set for shortest paths of a graph $G=(V,E)$, and present 
an upper bound to its VC-dimension. We show that this upper bound is strict. We
use the range set and the bound in our algorithm for estimating the betweenness
centrality of vertices of the graph.

The range set $\range_G$ is defined on the set $\mathbb{S}_G$ of all shortest
paths between vertices of $G$. It contains, for each vertex $v\in V$, the set
$\mathcal{T}_v$  of shortest paths that go through $v$:
\[
\range_G = \{T_v ~:~ v\in V\}\enspace.
\]

\begin{lemma}\label{lem:vcdimuppbound}
  Given a graph $G=(V,E)$ with edge-diameter $\Delta_G$, the range set
  $\range_G$ associated to the shortest paths in $G$ has VC-dimension
  $\VC(\range_G)$ at most $\lfloor\log_2\Delta_G\rfloor+1$.
\end{lemma}

\begin{proof}
Let $k>\lfloor\log_2\Delta_G\rfloor+1$ and assume for the sake of contradiction
that $\VC(\range_G)=k$. From the definition of the VC-dimension there is a set
$Q\subseteq\mathbb{S}_G$ of size $k$ that is shattered by $\range_G$. Let $p$ be
an element of $Q$. The path $p$ is contained in $2^{k-1}$ non-empty subsets of
$Q$. Let us label this non-empty subsets of $Q$ containing $p$ as
$S_1,\dotsc,S_{2^{k-1}}$, where the labelling is arbitrary.
Given that $Q$ is shattered then, for each set $S_i$ there must be a range $R_i$ in
$\range_G$ such that $S_i=Q\cap R_i$. Since all the $S_i$'s are
different from each other, then all the $R_i$'s must be different from each
other. Given that $p$ belongs to each $S_i$, then $p$ must also belong to each
$R_i$, that is, there are $2^{k-1}$ distinct ranges in $\range_G$ containing
$p$. But $p$ belongs to the ranges corresponding to the vertices through which $p$
passes, including the start and end nodes. This means that the number of ranges
in $\range_G$ that $p$ belongs to is equal to $|p|+1$, viz.~to the number of
edges in $p$ plus one. But $|p|\le\Delta_G$, by definition of $\Delta_G$, so $p$
can belong to at most $\Delta_G+1$ ranges from $\range_G$. Given that
$2^{k-1}>\Delta_G+1$, we reached a contradiction and there cannot be $2^{k-1}$
distinct ranges containing $p$, hence not all the sets $S_i$ can be expressed as
$Q\cap R_i$ for some $R_i\in\range_G$, but then $Q$ cannot be shattered and
$\VC(\range_G)\le\lfloor\log_2\Delta_G\rfloor+1$.%\qed
\end{proof}

In the restricted case when every pair of vertices has one or no shortest path
between them, the upper bound can be improved.
\begin{lemma}\label{lem:vcdimuppboundunique}
  Given a graph $G=(V,E)$ such that $|\mathcal{S}_{uv}|\le1$ for all pairs
  $(u,v)\in V\times V$, the range set $\range_G$ associated to the shortest
  paths in $G$ has VC-Dimension $\VC(\range_G)$ at most $2$.
\end{lemma}

\begin{proof}
  \XXX TBD \MR
\end{proof}

The bound presented in Lemma~\ref{lem:vcdimuppbound} is strict in the sense that
for each $d>0$ we can build a graph $G_d$ with edge-diameter $d$ and such
that the range set associated to the set of shortest path of $G_d$ has
VC-dimension exactly $\lfloor\log_2(d+1)\rfloor+1$.

\begin{lemma}
  Given $d>0$ there exists a graph $G$ with edge-diameter $d$ and such that
  $\VC(\range_G)=\lfloor\log_2(d+1)\rfloor+1$.
\end{lemma}

\begin{proof}
  \XXX TBD Remember to take care of the case $d=1$.
\end{proof}

