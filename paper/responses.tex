\documentclass{letter}
\nofiles
\usepackage{hyperref}
\usepackage{fullpage}
\usepackage{mdwlist}
\signature{\vspace{-40pt}
Matteo Riondato and\\Evgenios M. Kornaropoulos}
\address{Department of Computer Science \\ Brown University \\ Providence, RI 02912} %\\ USA}
\begin{document}
\begin{letter}{To: Tina Eliassi-Rad, Action Editor,\\%Data Mining and Knowledge Discovery,\\ 
	\qquad The Reviewers}
\opening{Dear Action Editor and Reviewers,}
We thank you for your service to the community, for the time you took to review
our paper, and for the insightful comments in your reviews: they have been very
valuable for improving our manuscript. In the following, we list how we
addressed all the suggested changes in our revised submission. 

\begin{enumerate*}
	\item {\bf Add the references mentioned by Reviewer \#1 and include a discussion
		of these works.}\\
		We added a paragraph to Section 2, discussing the suggested references,
		except for the one by Ugander et al., as we fail to see the connection
		with our work.
	\item {\bf Make the stylistic modifications mentioned by Reviewer \#2.}\\
		We modified the text to avoid the use of ``lazy'' citations. 

		We improved the style of the figures where we report the experimental
		results.
	\item {\bf Run and report on BP and top-k experiments mentioned by Reviewer
		\#2.} \\
		We run experiments for the accuracy of BP (experiments for the runtime
		were already presented) and added comments about them.

		We also run experiments to test the top-k algorithms, adding a
		subsection in the Experimental evaluation section.
	\item {\bf Make your code publicly available (as mentioned by Reviewer
		\#2).}\\
		Our code is available from
		\url{http://cs.brown.edu/~matteo/centrsampl.tar.bz2}. This URL is
		mentioned in the footnote 6. An independent implementation of our
		algorithms is also available as part of graphKIT
		(\url{https://networkit.iti.kit.edu/}).
	\item {\bf Elaborate on the definition of k-bounded-distance betweenness (as
		mentioned by Reviewer \#3).}\\
		We clarified the definition by making it more explicit and by
		reformulating ambiguous portions.
	\item {\bf Make the writeup in Section 6 less redundant (as mentioned by Reviewer
		\#3).}\\
		We rewrote the introduction to Section 6, and removed one of the
		presented variants. We believe that the variants included in the revised
		submission should stay, as they may either not be
		straightforward, as is the case for $k$-path betweenness, or are
		particularly interesting, as is the case for edge betweenness (an
		anonymous reviewer of the conference version asked us to explicitly
		mention edge betweenness).
	\item {\bf Elaborate on the definition of antichains (as mentioned by Reviewer
		\#3).}\\
		We rewrote the paragraph using the definition suggested by the Reviewer.
	\item {\bf Elaborate on the captions for Figures 8, 9, and 10 (as mentioned by
		Reviewer \#3).}\\
		We expanded the captions, including additional details regarding the
		reported quantities.
\end{enumerate*}

We hope to have addressed your comments to your satisfaction, and we thank you
in advance for any additional comment.
\closing{Sincerely,}
\end{letter}
\end{document}
