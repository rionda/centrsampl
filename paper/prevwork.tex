\section{Related work}\label{sec:prevwork}
Over the years, a number of centrality measures have been defined~\citep{Newman10}. In this work
we focus on betweenness centrality and some of its variants. 

Betweenness centrality was introduced in the sociology
literature~\citep{Anthonisse71,Freeman77}. \citet{Brandes08} presents a number
of minor variants. A particularly interesting one, $k$-bounded-distance
betweenness, limits the length of the shortest paths considered when computing
the centrality~\citep{BorgattiE06,Brandes08,PfefferC12}. This is not to be
confused with $k$-path betweenness centrality defined
by~\citet{KourtellisASIT12}, which considers simple random walks that are not
necessarily shortest paths. \citet{DolevEP10} present a generalization of
betweenness centrality which takes into account routing policies in the network. %
\ifproof
\citet{OpsahlAS10} define a new distance function between pair of vertices in
order to penalize paths with a high number of hops in weighted network. This
function induces a generalized and parametrized definition of betweenness.
\fi

The need of fast algorithms to compute the betweenness of vertices in a graph
arose as large online social networks started to appear. \citet{Brandes01} presents the
first efficient algorithm for the task, running in time $O(nm)$ on
unweighted graphs and $O(nm+n^2\log n)$ on weighted ones. The
algorithm computes, for each vertex $v$, the shortest path to every other vertex
and then traverses these paths backwards to efficiently compute the contribution
of the shortest paths from $v$ to the betweenness of other vertices. For very
large networks, the cost of this algorithm would still be prohibitive in
practice, so many approximation algorithms were
developed~\citep{JacobKLPT05,BrandesP07,BaderKMM07,GeisbergerSS08,MaiyaBW10,LimMRTB11}.
The use of random sampling was one of the more natural approaches to speed up
the computation of betweenness. Inspired by the work of~\citet{EppsteinW04},
\citet{JacobKLPT05} and independently \citet{BrandesP07} present an algorithm
that mimics the exact one, with the difference that,
instead of computing the contribution of all vertices to the betweenness of the
others, it only considers the contributions of some vertices sampled uniformly
at random. To guarantee that all estimates are within $\varepsilon$ from their
real value with probability at least $1-\delta$, the algorithm
from~\citep{JacobKLPT05,BrandesP07} needs $O(\log(n/\delta)/\varepsilon^2)$
samples. The analysis for the derivation of the sample size uses Hoeffding bounds~\citep{Hoeffding63} 
and the union bound~\citep{MitzenmacherU05}. 
\citet{GeisbergerSS08} noticed that this can lead
to an overestimation of the betweenness of vertices that are close to the
sampled ones and introduced different unbiased estimators that are
experimentally shown to have smaller variance and do not suffer from this
overshooting. Our algorithm is different from these because we sample, each
time, a single random shortest path. %between two randomly chosen nodes. 
This leads to a much smaller sample size and less work done for each sample,
resulting in a much faster way to compute approximations of the betweenness with
the same probabilistic guarantees. 
\ifproof
Although existing algorithms \citep{BrandesP07,JacobKLPT05,GeisbergerSS08} can be extended to return a
superset of the top-$K$ vertices with highest betweenness, they only offer an
\emph{additive} approximation guarantee, while our algorithm for the top-$K$
vertices offers a \emph{multiplicative} factor guarantee, which is much
stricter. 
\fi
We delve more in the comparisons with these algorithms in
Sect.~\ref{sec:discussion} and~\ref{sec:exper}. 

A number of works explored
the use of adaptive sampling, in contrast with the previous algorithms (and
ours) which use a fixed sample size. \citet{BaderKMM07} present an adaptive
sampling algorithm which computes good estimations for the betweenness of
high-centrality vertices, by keeping track of the partial contribution of each
sampled vertex, obtained by performing a single-source shortest paths
computation to all other vertices. \citet{MaiyaBW10} use concepts from expander
graphs to select a connected sample of vertices. They estimate the betweenness
from the sample, which includes the vertices with high centrality. They build
the connected sample by adding the vertex which maximizes the number of
connections with vertices not already in the sample. Modified versions
of this algorithm and an extensive experimental evaluation appeared
in~\citep{LimMRTB11}. The algorithm does not offer any guarantee on the quality
of the approximations. Compared to these adaptive sampling approaches, our
methods ensure that the betweenness of all (or top-$K$) vertices is well
approximated, while using a fixed, predetermined amount of samples.
\citet{SaryuceSKC13} present an algorithm that pre-processes the network in
multiple ways by removing degree-1 vertices and identical vertices, and splits it
in separate components where the computation of betweenness can be performed
independently and then aggregated. They do not present an analysis of the
complexity of the algorithm. 

In the analysis of our algorithm we use results from VC-dimension
theory~\citep{VapnikC71}, a key component of statistical learning theory. We
compute an upper bound to the VC-dimension of a range set defined on shortest
paths. \citet{KranakisKRUW97} present a number of results on the VC-dimension of
various range sets for graphs (stars, connected sets of vertices, sets of
edges), but do not deal with shortest paths. \citet{AbrahamDFGW11} use
VC-dimension to speed up shortest path computation but their range set is
different from the one we use: their ground set is the set of vertices while
ours is defined on shortest paths.

\ifproof
The present work extends substantially the preliminary
version~\citep{RiondatoK14WSDM}. The proof of all the theorems and lemmas are
presented here for the first time. Examples of betweenness centrality and of the
concept of VC-dimension have been added to Sect.~\ref{sec:prelims}. A major new
contribution is a tighter analysis of the algorithm from~\citep{BrandesP07},
presented in Sect.~\ref{sec:discussion}. We also discuss many other variants of
betweenness, including edge betweenness in Sect.~\ref{sec:variants}, where we
also present a new tighter analysis of the algorithm for $k$-path betweenness
centrality by~\citep{KourtellisASIT12}.  %An expanded
%motivation for the top-$k$ algorithm opens Sect.~\ref{sec:topk}. 
%Finally, we report many additional
%results of our experimental evaluation, including results on the top-$k$
%algorithm, in Sect.~\ref{sec:exper}.
\fi

