\section{Related work}\label{sec:prevwork}
Over the years, a number of centrality measures have been defined. In this work
we are only concerned with betweenness centrality and some of its variants. We
refer the reader interested in other measure to the book by~\citet{Newman10} and
references therein.

Betweenness centrality was introduced in the sociology
literature~\citep{Anthonisse71,Freeman77}. 

Variants~\citep{Brandes08,DolevEP10,KourtellisASIT12,PfefferC12}

The need of fast algorithms to compute
the betweenness of vertices in a graph arose as large online social networks
developed. \citet{Brandes01} presents the first efficient algorithm for the
task, running in time $O(|V|\cdot|E|)$ on unweighted graphs and
$O(|V|\cdot|E|+|V|^2\log|V|)$ on weighted ones. The algorithm computes, for each
vertex $v$, the shortest path to every other vertex and then traverses these paths
backwards to efficiently compute the contribution of the shortest paths from $v$
to the betweenness of other vertices. For very large networks, the cost of this
algorithm would still be prohibitive in practice, so many approximation
algorithms were
developed~\citep{JacobKLPT05,BrandesP07,BaderKMM07,GeisbergerSS08,MaiyaBW10,LimMRT11}.
The use of random sampling was one of the more natural approaches to speed up
the computation of betweenness. Inspired by the work of~\citet{EppsteinW04},
\citet{JacobKLPT05,BrandesP07} present an algorithm that samples vertices at
random and compute the contribution of the sampled vertices to the betweenness
of every vertex. The number of samples needed to guarantee that, with
probability at least $1-\delta$, the estimate for each node is within
$\binom{|V|}{2}\varepsilon$ from the real value, for some $\varepsilon\in(0,1)$,
is $O(\log(n/\delta)/\varepsilon^2)$. The analysis uses Hoffeding bounds and the
union bound. The algorithm mimics the exact one, with the difference
that, instead of computing the contribution of all vertices to the betweenness
of the others, only it only consider the contribution of the sampled vertices.
\citet{GeisbergerSS08} noticed that this can lead to an overestimation of the
betweenness of vertices that are close to the sampled ones and introduced
different unbiased estimators that are experimentally shown to have smaller
variance and do not occur in this overshooting. Our algorithm is different
because each sample in our case is a single random shortest path between two
randomly chosen nodes. This leads to a much smaller sample size and less work
done for each sample, resulting in a much faster way to compute approximations
of the betweenness with the same probabilistic guarantees. We delve more in the
comparisons with these algorithm in Sect.~\ref{sec:discussion}
and~\ref{sec:exper}.
A number of works explored the use of adaptive sampling, in contrast with the
previous algorithms (and ours) which use a fixed sample size. \citet{BaderKMM07}
present an adaptive sampling algorithm which computes good estimations for the
betweenness of high-centrality vertices, by keeping track of the partial
contribution of each sampled vertex, obtained by performing a single-source
shortest paths computation to all other vertices. \citet{MaiyaBW10} use concepts
from expander graphs to select a connected sample of vertices which includes the
highest centrality ones and such that the betweenness computed on the induced
sample subgraph is close to the real one. They achieve these by repeatedly
including in the sample the vertex in the neighborhood of the sample which
maximizes the number of connections with vertices not already in the sample.
Modified versions of this algorithm and an extensive experimental evaluation
appeared in~\citep{LimMRTB11}. The algorithm does not offer any guarantee on the
quality of the approximations. Compared to these adaptive sampling approaches,
our method ensures that the betweenness of all vertices is well approximated
with a fixed, predetermined amount of samples.

Other approximations~\citep{GkorouPE10,PrountzosP13,SaryuceSKC13}

VC-Dimension on graphs~\citep{AnthonyBC95,KranakisKRUW97,MubayiZ07,YcartR07}
(mostly only~\citep{KranakisKRUW97}

VC-Dimension and Shortest paths~\citep{AbrahamDFGW11}

