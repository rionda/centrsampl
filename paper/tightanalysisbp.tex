\documentclass{article}
\usepackage{dsfont}
\usepackage{fullpage}
\usepackage{amssymb,amsmath,amsthm}
\usepackage{datetime}
\usepackage{natbib}
\newtheorem{lemma}{Lemma}
\def\betw{\mathsf{b}}
\def\XXX{{\bf XXX}}
\def\MR{{\bf MR}}
\def\EK{{\bf EK}}
\title{}
\author{}
\begin{document}
%\maketitle
{\bf \large \today, \currenttime}

\section{Tighter Analysis for the BP Algorithm}
The analysis of Algorithm 1 allow us to obtain a tighter analysis for the
algorithm by~\citet{BrandesP07}.

\begin{lemma}
  If Algorithm 1 computes an $(\varepsilon,\delta)$-approximation with $r$
  samples, then the algorithm by~\citep{BrandesP07} does the same with the same
  number of samples.
\end{lemma}

\begin{proof}
Consider a simple coupling of the executions of the two algorithms. When
the Algorithm 1 samples a shortest path from a vertex $v$ to a vertex $u$, the
algorithm by~\citep{BrandesP07} samples the vertex $v$.

Let $X_v$ be the number of times that vertex $v$ is the starting point of a
sampled path. 

The estimation for the betweenness of vertex $w$ computed by the algorithm
by\citep{BrandesP07} is
\[
\tilde\betw_{\mathrm{BP}}(w)=\frac{1}{r}\sum_{v\in V}
X_v\underbrace{\left(\frac{1}{n-1}\sum_{\substack{u\in V \\u\neq
v}}\sum_{p\in\mathcal{S}_{uv}}\frac{\mathds{1}_{\mathsf{Int}(p)}(w)}{|\mathcal{S}_{u,v}|}\right)}_{Y_v(w)}\enspace.
\]
The estimation for the betweenness of vertex $w$ computed by Algorithm 1
is
\[
\tilde\betw_{\mathrm{RK}}(w)=\frac{1}{r}Z_w,
\]
where $Z_w$ is the number of times that a path containing the vertex $w$ gets
sampled. Now, for each $v$ and for $1\le i\le X_v$, let $p_v^{(i)}$ be the path
sampled the $i$\textsuperscript{th} time that the sampled path had $v$ as
starting vertex, and let $A_v^{(i)}(w)=1$ if $w\in\mathsf{Int}(p_v^{(i)}$, and
$A_v^{(i)}(w)=0$ otherwise. Then it is clear that 
\[
\tilde\betw_{\mathrm{RK}}(w)=\frac{1}{r}\sum_{v\in V}
X_v\frac{\sum_{i=1}^{X_v}A^{(i)}_v(w)}{X_v}\enspace.
\]
We claim that
\[
\mathbb{E}\left[\left.\frac{\sum_{i=1}^{X_v}A^{(i)}_v(w)}{X_v}\right|X_v\right]=Y_v(w)\enspace.\]
To see this, consider $\mathbb{E}\left[\left.A_v^{(i)}(w)\right|X_v\right]$. It
is clear that $A^{(i)}(w)$ and $X_v$ are independent, for $1\le i\le X_v$. By applying the definition of probability we have 
\[
\mathbb{E}\left[\left.A_v^{(i)}(w)\right|X_v\right] =
\mathbb{E}\left[A_v^{(i)}(w)\right]=\frac{1}{n-1}\sum_{\substack{u\in V
\\u\neq v}}\sum_{p\in\mathcal{S}_{uv}}\frac{\mathds{1}_{\mathsf{Int}(p)}(w)}{|\mathcal{S}_{u,v}|}
=Y_v(w)\enspace.
\]
This expression does not depend on $i$.

The claim then follows easily from the linearity of expectation:
\[
\mathbb{E}\left[\left.\frac{\sum_{i=1}^{X_v}A^{(i)}_v(w)}{X_v}\right|X_v\right] =
\sum_{i=1}^{X_v}\mathbb{E}\left[\left.\frac{A^{(i)}_v(w)}{X_v}\right|X_v\right]=\sum_{i=1}^{X_v}\frac{1}{X_v}\mathbb{E}\left[A_v^{(i)}(w)\right]=\mathbb{E}\left[A_v^{(i)}(w)\right]=Y_v(w)\enspace.
\]
This implies that the estimator $\tilde\betw_{\mathrm{BP}}(w)$ has lower variance than the estimator
$\tilde\betw_{\mathrm{RK}}(w)$, for all vertices $w$. Since both estimators have the same
expectation $\betw(w)$, this means that for any $a\in[0,1]$ we have
\[
\Pr\left(|\tilde\betw_{\mathrm{BP}}(w)-\betw(w)|>a\right)\le\Pr(\left(|\tilde\betw_{\mathrm{RK}}(w)-\betw(w)|>a\right),
\]
which implies the thesis.

\end{proof}

The estimator $\tilde\betw_{\mathrm{BP}}(w)$ has lower variance than
$\tilde\betw_{\mathrm{RK}}(w)$ but before always opting for the algorithm
by~\citet{BrandesP07}, one should take into account the fact that, per sample,
the computation of $\tilde\betw_{\mathrm{BP}}(w)$ requires more time than the
one for $\tilde\betw_{\mathrm{RK}}(w)$. The difference is even larger if
Algorithm 1 uses bidirectional search~\citep{KaindlK97,Pohl69}. We can then see
that Algorithm 1 and the algorithm by~\citet{BrandesP07} share the same design
principles, but choose different trade-offs between accuracy and speed.

\MR: my intuition is that we can actually analyze the algorithm
by~\citet{BrandesP07} using the VC-dimension and showing that what they do is an
ergodic sampling. Recent results show that the VC-dimension can be used when the
sampling process is ergodic~\citep{AdamsN10}.

\bibliographystyle{abbrvnat}
\bibliography{bidirectionalsearch,diamapprox,vcmine,vcgraph,centrality,various}

\end{document}

