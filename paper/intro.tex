\section{Introduction}\label{sec:intro}
\begin{itemize}
  \item Centrality indices are a fundamental tool for network analysis
  \item They measure the relative importance of a vertex in the network.
  \item A lot of them are based on shortest path computation. Betweenness
    centrality. Used for social networks, protein network, network traffic in
    communication networks, highway nodes identification. (see Kourtellis for references)
  \item thanks to the amount of data collection and the rise of online social
    networks, we now have much larger networks than in the past
  \item Analyzing them is a challenge due to their very large size.
  \item There are polynomial time algorithms for computing many centrality
    indices, but they are not practical. 
  \item There is a need for more efficient algorithms. Given that data mining is
    exploratory in nature, approximate results can be sufficient, especially if
    the approximation factor is guaranteed to be within user-specified limits.
    In practice, more than the actual value of the betweenness of each vertex,
    the user is interested in the relative ranking of the node.
\end{itemize}

\paragraph{Our contributions} 
We present two randomized algorithms to approximate the betweenness centrality
(and some of its variants) of vertices of a graph. The first algorithm
guarantees that the estimated betweenness values for all vertices are within an
\emph{additive} factor $\varepsilon$ from the real values, with probability at
least $1-\delta$. The second algorithm focuses on the top-$K$ vertices with
highest betweenness and returns a set of vertices which includes the top-$K$,
while ensuring that the estimated betweenness for all returned vertices is
within a \emph{multiplicative} factor $\varepsilon$ from the real value, with
probability at least $1-\delta$. This is the first algorithm to reach such a
high-quality approximation for the set of top-$K$ vertices. The algorithms are
based on random sampling of shortest paths. The analysis to derive the
sufficient sample size uses notions and tools from VC-Dimension theory. This
results in a much smaller sample size than that used by other algorithms
guaranteeing the same approximation. Moreover, the amount of work performed by
our algorithms per sample is also less than what was done in previous works.
We extensively evaluated our methods on real graphs and compared their
performances to the exact algorithm for betweenness
centrality~\citep{Brandes01} and to other sampling-based approximation
algorithms~\citep{JacobKLPT05,BrandesP07}.

