\section{Introduction}\label{sec:intro}
\XXX TBD

\paragraph{Our contributions} 
We present two randomized algorithms to approximate the betweenness centrality
of vertices of a graph. The first algorithm guarantees that the estimated betweenness
values for all vertices are within an \emph{additive} factor $\varepsilon$ from the
real values, with probability at least $1-\delta$. The second algorithm focuses
on the top-$K$ vertices with highest betweenness and returns a set of vertices
which includes the top-$K$, while ensuring that the estimated betweenness for
all returned vertices is within a \emph{multiplicative} factor $\varepsilon$
from the real value, with probability at least $1-\delta$. This is the first
algorithm to reach such a high-quality approximation for the set of top-$K$
vertices. The algorithms are based on random sampling of shortest paths. The
analysis to derive the sufficient sample size uses notions and tools from
VC-Dimension theory. This results in a much smaller sample size than that used
by other algorithms guaranteeing the same approximation. Moreover, the amount of
work performed by our algorithms per sample is also less than what was done in
previous works.  We extensively evaluated our methods on real graphs and
compared its performances to the exact algorithm for betweenness
centrality~\citep{Brandes01} and to other sampling based
algorithms~\citep{BrandesP07,JacobKLPT05}.


