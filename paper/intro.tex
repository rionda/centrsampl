\section{Introduction}\label{sec:intro}
Centrality indices are fundamental metrics for network analysis. They express the
relative importance of a vertex in the network. Some of them, e.g., degree
centrality, reflect local properties of the underlying graph, while others,
like betweenness centrality, give information about the global network
structure, as they are based on shortest path computation and counting~\citep{Newman10}. In
this work we are interested in \emph{betweenness
centrality}~\citep{Anthonisse71,Freeman77}, that is, for every vertex in the
graph, the fraction of shortest paths that goes through that vertex (see
Section~\ref{sec:prelims} for formal definitions). Betweenness centrality has
been used to analyze social and protein interaction networks, to evaluate traffic in
communication networks, and to identify important intersections in road
networks~\citep{Newman10,GeisbergerSS08}. There exist polynomial-time
algorithms to compute the exact betweenness centrality~\citep{Brandes01}, but
they are not practical for the analysis of the very large networks that are of
interest these days. Graphs representing online social networks, communication
networks, and the web graph have millions of nodes and billions of edges,
making a polynomial-time algorithm too expensive in practice. Given that data
mining is exploratory in nature, approximate results are usually sufficient,
especially if the approximation error is guaranteed to be within user-specified
limits. In practice, the user is interested in the relative ranking of the
vertices according to their betweenness, rather than the actual value of the
betweenness, so a very good estimation of the value of each vertex is
sufficiently informative for most purposes. It is therefore natural to develop
algorithms that trade off accuracy for speed and efficiently compute
high-quality approximations of the betweenness of the vertices.  Nevertheless,
in order for these algorithms to be practical it is extremely important that
they scale well and have a low runtime dependency on the size of the network
(number of vertices and/or edges).

\paragraph{Our contributions} 
We present two randomized algorithms to approximate the betweenness centrality
(and some of its variants) of the vertices of a graph. The first algorithm
guarantees that the estimated betweenness values for all vertices are within an
\emph{additive} factor $\varepsilon$ from the real values, with probability at
least $1-\delta$. The second algorithm focuses on the top-$K$ vertices with
highest betweenness and returns a \emph{superset} of the top-$K$,
while ensuring that the estimated betweenness for all returned vertices is
within a \emph{multiplicative} factor $\varepsilon$ from the real value, with
probability at least $1-\delta$. This is the first algorithm to reach such a
high-quality approximation for the set of top-$K$ vertices. The algorithms are
based on random sampling of shortest paths. The analysis to derive the
sufficient sample size is novel and uses notions and results from VC-dimension
theory. We define a range set associated with the problem at hand and prove strict
bounds to its VC-dimension. The resulting sample size \emph{does not
depend on the size of the graph}, but only on the maximum number of vertices
in a shortest path, a \emph{characteristic quantity} of the graph that we call
the \emph{vertex-diameter}. For some networks, we show that the VC-dimension is
actually
at most a constant and so the sample size depends \emph{only on the approximation
parameters} and not on any property of the graph, a somewhat surprising fact
that points out interesting insights. Thanks to the lower runtime dependency on
the size of the network, our algorithms are \emph{much faster and more scalable}
than previous contributions~\citep{JacobKLPT05,BrandesP07,GeisbergerSS08}, while
offering the same approximation guarantees. Moreover, the amount of work
performed by our algorithms per sample is also less than that of others algorithms.
We extensively evaluated our methods on real graphs and compared their
performances to the exact algorithm for betweenness centrality~\citep{Brandes01}
and to other sampling-based approximation
algorithms~\citep{JacobKLPT05,BrandesP07,GeisbergerSS08}, showing that our
methods achieve a huge speedup (3 to 4 times faster) and scale much better as
the number of vertices in the network grows.

We present related work in Sect.~\ref{sec:prevwork}. Section~\ref{sec:prelims}
introduces all the basic definitions and results that we will use throughout the
paper. A range set for the problem at hand and the bounds to its VC-dimension
are presented in Sect.~\ref{sec:rangeset}. Based on these results we develop and
analyze algorithms for betweenness estimation that we present in
Sect.~\ref{sec:algo}. Section~\ref{sec:exper} reports the methodology and
the results of our extensive experimental evaluation.

