\ifdmkd
\else
\begin{abstract}
\fi
  %We present an efficient randomized algorithm for approximating the {\sl
  %betweenness centrality} of all nodes in a network within strict probabilistic
  %guarantees.
  Betweenness centrality is a fundamental measure in social network
  analysis, %, %quantifying
  expressing the importance or influence of individual vertices in a
  network in terms of the fraction of shortest paths that pass through them. 
  %The fastest known algorithm to
  %compute the exact betweenness of all vertices in a graph $G=(V,E)$ in
  %$O(|V|\cdot|E|)$ steps in the unweighted case and in
  %$O(|V|\cdot|E|+|V|^2\log|V|)$ steps in weighted graphs. 
  Exact computation in large networks %with millions of vertices and hundred of millions of edges 
  is prohibitively expensive and fast approximation algorithms are required in
  these cases. 
  We present two efficient randomized %approximation 
  algorithms for betweenness
  estimation. The algorithms are based on random sampling of shortest paths and
  offer probabilistic guarantees on the quality of the approximation. 
  %One algorithm computes an $(\varepsilon,\delta)$-approximation of the vector
  %of betweenness centrality of all the vertices in the graph.
  The first algorithm estimates the betweenness of all vertices: all
  approximate values are within an additive factor $\varepsilon\in(0,1)$ from the real
  values, with probability at least $1-\delta$. The second algorithm focuses on
  the top-K vertices with highest betweenness and estimate their betweenness
  value to within a multiplicative factor $\varepsilon$, with probability at least
  $1-\delta$. This is the first algorithm that can compute such approximation for
  the top-K vertices. We use results from the VC-dimension theory to develop
  bounds to the sample size needed to achieve the desired approximations. By
  proving upper and lower bounds to the VC-dimension of a range set associated
  with the problem at hand, we obtain a sample size that is independent from the
  number of vertices in the network and only depends on a characteristic
  quantity that we call the vertex-diameter, that is the maximum number of
  vertices in a shortest path. In some cases, the sample size is completely
  independent from any property of the graph.
  %Extensive experimental evaluation on real and artificial networks demonstrate
  %the practicality of our methods and their performances compared to exact and
  %approximated approximated algorithms.
  The extensive experimental evaluation that we performed using real and
  artificial networks shows that our algorithms are significantly faster and
  much more scalable as the number of vertices in the network grows than
  previously presented algorithms with similar approximation guarantees.
\ifdmkd
\else
\end{abstract}
\fi
  
