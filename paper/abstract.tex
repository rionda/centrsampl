\begin{abstract}
  %We present an efficient randomized algorithm for approximating the {\sl
  %betweenness centrality} of all nodes in a network within strict probabilistic
  %guarantees.
  Betweenness centrality is a fundamental measure in social network
  analysis, %, %quantifying
  expressing the importance or influence of individual vertices in a
  network in terms of the fraction of shortest paths that pass through them. 
  %The fastest known algorithm to
  %compute the exact betweenness of all vertices in a graph $G=(V,E)$ in
  %$O(|V|\cdot|E|)$ steps in the unweighted case and in
  %$O(|V|\cdot|E|+|V|^2\log|V|)$ steps in weighted graphs. 
  Exact computation in large networks %with millions of vertices and hundred of millions of edges 
  is prohibitively expensive and fast approximation algorithms are required in
  these cases. 
  We present two efficient randomized algorithms that offer probabilistic
  guarantees on the quality of the approximation. %One algorithm computes an
  %$(\varepsilon,\delta)$-approximation of the vector of betweenness centrality
  %of all the vertices in the graph.
  The first algorithm computes an approximation of the betweenness for all vertices. All 
  approximated values are within an additive factor $\varepsilon$ from the real values,  
  with probability at least $1-\delta$. The second algorithm focuses on the
  top-K vertices with highest betweenness and computes an approximation for the
  corresponding values, with all approximated values being within a multiplicative factor
  $\varepsilon$ from the real ones, with probability at least $1-\delta$. This
  is the first algorithm that reaches such approximation for the top-K vertices.
  Our algorithms are significantly faster than previously presented algorithms
  with similar approximation guarantees. We achieve this speedup through a
  novel application of the VC-dimension theory to optimize the sampling
  component of the algorithms: we prove upper and lower bounds to the
  VC-dimension of a range set associated to the problem at hand.
  Extensive experimental evaluation on real and artificial networks demonstrate
  the practicality of our methods and their performances compared to exact and
  approximated approximated algorithms.
  \XXX Too many repetitions of ``approximation''. Mention scalability and work.
\end{abstract}
  
