\begin{abstract}
  Betweenness centrality is an important measure in (social) network analysis as
  it allows to rank the vertices by their ``prominence'' in the network. The
  betweenness of a vertex $v$ is, roughly, the number of shortest paths among
  vertices of the network that go through $v$. The currently fastest algorithm
  to compute the exact betweenness of all vertices in a graph
  $G=(V,E)$ requires time $O(|V|\cdot|E|)$ if the edges have no weight and time
  $O(|V|\cdot|E|+|V|^2\log|V|)$ for weighted graphs. Exact computation in large
  networks with millions of vertices and hundred of millions is therefore
  prohibitively expensive and fast approximation algorithms are required in
  these cases. In this work we develop an algorithm based on random sampling
  that probabilistically guarantees that the estimated betweenness of each
  vertex in the graph is very close to its real value. Each sample is a random
  shortest paths between two randomly chosen vertices of the graph. The required
  sample size depends on the desired accuracy $\varepsilon$ and confidence
  $\delta$ and on a characteristic quantity $\delta$ of the graph that we call
  \emph{vertex-diameter}, viz.~the maximum number of vertices in a shortest
  path. The resulting sample size $|S|=O(\log(\Delta/\delta)/\epsilon^2)$ is a
  huge improvement over existing algorithms offering the same guarantees. In the
  experimental evaluation on real and artificial networks we compare with these
  existing methods and show that our algorithm is much faster.
\end{abstract}

